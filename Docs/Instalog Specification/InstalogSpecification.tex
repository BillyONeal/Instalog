\documentclass[letterpaper,12pt]{article}
\usepackage[margin=1in]{geometry}
\usepackage{xltxtra}
\setmainfont[Mapping=tex-text]{Liberation Serif}
\setmonofont[Scale=0.8]{Liberation Mono}
\newcommand{\var}[1]{\texttt{\$\{#1\}}}
\usepackage[colorlinks=false,pdfborder=0 0 0]{hyperref}
\usepackage{graphicx}

\title{Instalog Requirements Specification}
\author{
Billy R. O'Neal III (bro4@case.edu) \\
Jacob Snyder (jrs213@case.edu)
}

\begin{document}

\maketitle

\section{Introduction}
Instalog is a senior project by Jacob Snyder and Billy O'Neal, which is designed
to gather information from Microsoft Windows installations, for the purpose of
malware removal and system repair.  It must generate a human and machine
readable report which assists end users, remote experts, and local
administrators with issue diagnosis and malware removal.

Instalog is inspired by several similar tools which all share some basic
functionality.  In many ways, Instalog can be viewed as an evolution of
these tools:
\begin{itemize}
    \item TrendMicro's {\em Hijack This} (HJT)
    \item ``sUBs'' {\em Doesn't Do Squat} (DDS)
    \item ``random/random'''s {\em Random's System Information Tool} (RSIT)
    \item ``OldTimer'''s {\em OTA}, {\em OTS}, and {\em OTL} (formerly
    OTAnalyzeIt, OTScanIt, and OTListIt, respectively)
    \item Sysinternals' {\em Autoruns}
    \item Runscanner's {\em Runscanner}
\end{itemize}
all of which purport to accomplish similar goals to Instalog. However, each of
these tools has bugs or specific behavior which cause problems for at least one
of Instalog's three intended user groups.

Specifically, the above tools contain one or more of the above problems:
\begin{itemize}
    \item Incorrect handling and escaping of log data
    \item Lack of published specifications, documentation, or source code
    \item Outstanding bugs that the authors are unwilling or unable to fix
    \item Lack of scriptability, for the purposes of modifying log output and
    malware removal.
    \item Lack of 64 bit support.
    \item Lack of Unicode support.
    \item Lack of enumeration of some types of useful log information.
\end{itemize}
Instalog will attempt to solve those problems by combining characteristics of
the above tools which are deemed useful, while mixing in a few tricks of it's
own.

\subsection{Document Conventions}
Within the scope of this document, computer output or other information that is
to be taken literally is written in \texttt{fixed width text}. In some
instances, a block of fixed width text will be surrounded by non-fixed width
quotes ``\texttt{ like this}''. In such cases, the quotes are not significant,
and there will there will (typically) be leading or trailing space around the
fixed width block, which is significant and MUST NOT be removed. Variables,
which are replaced with some content, are written of the form \var{name}, and
will be explained in greater detail in prose surrounding a given block of
\texttt{fixed width} text.

Portions of this document are written in terms of the \verb|.REG| format
described by Regedit, a windows component. The syntax for the regedit script is
described in several locations, in particular  of \texttt{monospaced}
(\url{http://en.wikipedia.org/wiki/Windows_Registry#.REG_files}).

Additionally WOW64 (Windows on Windows 64) defines a system whereby registry
virtualization is in effect -- 32 bit programs are shown a virtualized view of
the 64 bit registry. Current versions of windows implement this by putting the
32 bit registry view in a key called ``\verb|Wow6432Node|'' in the root of each
hive in the registry. Additionally, Windows provides flags to the functions that open
registry keys, such as \verb|NtOpenKey| or \verb|RegCreateKeyEx|, which select
the correct view. Microsoft asks that applications be built in terms of the API
flags, because the specific name of \verb|Wow6432Node| is an implementation
detail which can change in future versions of Windows.

Therefore, we will define on top of the \verb|.REG| format a comment of
\verb|;32 Bit Registry View| or \verb|;64 Bit Registry View|, which represent
the flag that would be passed into Windows when opening the indicated key.

\subsection{Intended Audience}
Instalog is designed with three types of target users in mind. These ``user
classes'' are listed in the following sections.

\subsubsection{Home Users}
For a typical home user, Instalog must not display a complicated interface, and
must make it relatively difficult to misstep and take a wrong action. Few
options need be presented, such as the ability to generate a default report and
the ability to take a given script and run it on a target machine. Complicated
features such as analysis must not be displayed; though they may appear as
options that are, by default, deselected.

\subsubsection{Administrators}
Administrators are similar to home users in that they are physically working at
a computer being examined, but they are different in that they have the intent of
repairing their own computer or the computer of a client. They wish to see
analysis features and more possible options. Instalog must provide a means for
Administrators to use it's analysis features without manual saving and reloading
of log files.

\subsubsection{Forum Experts}
Forum Experts help typical end users repair their machines remotely over
self-help forums such as BleepingComputer.com or GeeksToGo.com.
These users work remotely, and likely will never see a given target
machine.
Instalog must produce log formats that are human readable in the vast majority
of cases, but which can be passed through common forum software such as Invision
Power Board, phpBB, or vBulletin without destruction of information.
Unfortunately, this makes common data exchange formats such as JSON and XML
unsuitable. 

Moreover, as obtaining additional information from a machine may
have lead times of several days, Instalog's report must be unambiguous; that is,
no two possible system configurations may produce the same output. Experts can
also benefit from log analysis features. Finally, Experts need to be able to
write simple, human readable scripts to perform actions to fix a user's machine
remotely.

\subsection{Acknowledgements}
Instalog's authors would like to thank ``sUBs'' for use of DDS's whitelisting
data for use in Instalog, and for being available for occasional clarification
of problems. He also allowed use of a modified form of DDS' logging format.

Instalog also was constructed with feedback taken from self help forums like
BleepingComputer, and students in Dr. Glutekin Ozsoyoglu's EECS 395: Senior
Project class of Spring 2012, at Case Western Reserve University.

\subsection{Licensing} \label{sec:licensing}
Instalog itself is to be released under the two clause form of the BSD license,
which is reprinted below:

\begin{verbatim}
Copyright © 2012, Jacob Snyder, Billy O'Neal III, and "sUBs"
All rights reserved.

Redistribution and use in source and binary forms, with or without
modification, are permitted provided that the following conditions are met: 

1. Redistributions of source code must retain the above copyright notice, this
   list of conditions and the following disclaimer. 
2. Redistributions in binary form must reproduce the above copyright notice,
   this list of conditions and the following disclaimer in the documentation
   and/or other materials provided with the distribution. 

THIS SOFTWARE IS PROVIDED BY THE COPYRIGHT HOLDERS AND CONTRIBUTORS "AS IS" AND
ANY EXPRESS OR IMPLIED WARRANTIES, INCLUDING, BUT NOT LIMITED TO, THE IMPLIED
WARRANTIES OF MERCHANTABILITY AND FITNESS FOR A PARTICULAR PURPOSE ARE
DISCLAIMED. IN NO EVENT SHALL THE COPYRIGHT OWNER OR CONTRIBUTORS BE LIABLE FOR
ANY DIRECT, INDIRECT, INCIDENTAL, SPECIAL, EXEMPLARY, OR CONSEQUENTIAL DAMAGES
(INCLUDING, BUT NOT LIMITED TO, PROCUREMENT OF SUBSTITUTE GOODS OR SERVICES;
LOSS OF USE, DATA, OR PROFITS; OR BUSINESS INTERRUPTION) HOWEVER CAUSED AND
ON ANY THEORY OF LIABILITY, WHETHER IN CONTRACT, STRICT LIABILITY, OR TORT
(INCLUDING NEGLIGENCE OR OTHERWISE) ARISING IN ANY WAY OUT OF THE USE OF THIS
SOFTWARE, EVEN IF ADVISED OF THE POSSIBILITY OF SUCH DAMAGE.
\end{verbatim}

This document, along with all other documentation related to Instalog,  is to be
released under the Creative Commons Attribution 3.0 Unported license. Human
readable and lawyer readable versions of this license can be found at
\url{http://creativecommons.org/licenses/by/3.0/}.

\subsection{Minimum System Requirements}
Instalog MUST run on all Microsoft Windows NT variants released later than
Windows 2000 SP4:
\begin{itemize}
  \item Windows 2000 (x86, SP4 only)
  \item Windows XP (x86 and x64, RTM, SP1, SP2, and SP3)
  \item Windows Vista (x86 and x64, RTM, SP1, and SP2)
  \item Windows 7 (x86 and x64, RTM and SP1)
  \item Windows Server 2003 (x86 and x64, RTM, SP1, and SP2)
  \item Windows Server 2003 R2 (x86 and x64, RTM, SP1, and SP2)
  \item Windows Server 2008 (x86 and x64, RTM, SP1, and SP2)
  \item Windows Server 2008 R2 (x64, RTM, and SP1)
\end{itemize}

No attempt will be made to support Itanium architecture systems as Instalog's
authors do not have access to suitable testing hardware. No attempt will be made
to support MS-DOS based versions of Windows. Instalog's behavior on unsupported
machines must not cause data destruction, but is otherwise undefined.

\subsection{Scope Limit}
While the authors would like to complete this entire tool before the completion
of the Senior Project class in April, they realize that this may not be possible
given the scale of the tool. Therefore, if they must cut features to meet the
deadline, these features shall be cut, in order (lower number on this list
means more likely to be cut):
\begin{enumerate}
    \item Checkboxes and fix generation on the GUI
    \item ``Value Added'' scripting actions, such as MRC upload or VirusTotal
    upload
    \item Spoofed DNS check
    \item Enumeration of Google Chrome data
    \item Enumeration of Firefox data
    \item GUI altogether
    \item Scripting altogether
\end{enumerate}

In the event any of these features are removed, Instalog must remain a
functional, shippable, well tested, production-quality product, suitable for the
author's Senior Project submission.

\section{Graphical User Interface}
The Graphical User Interface (GUI) is designed in such a way that it can
accommodate the various usage scenarios common for this tool.  Therefore, it
must bridge the gap between simplicity and complexity so that users can simply
use the tool and power users can use the tool to create powerful system-altering
scripts.

The GUI is inspired by the well-known Windows application installer paradigm. 
Basic users will only encounter screens similar to what they are familiar with
when installing or uninstalling applications.  The interface will become much
more complex when power users use the tool to modify scripts, but this is to be
expected.  The editing script interface is inspired by tools that exist in the
field (namely OTA and HijackThis).

\subsection{Main}
When a user opens the application, they must be presented with the screen from
Figure~\ref{fig:gui_main}.  This screen is the decision point of the
application.  Depending on the user's input, this screen will take the user
through the workflows of this tool.

\begin{figure}[h]
  	\centering
	\includegraphics{figures/gui/Main.png}
  	\caption{GUI Main Screen}
  	\label{fig:gui_main}
\end{figure}

\begin{description}
\item[Textbox requirements] \hfill
\begin{enumerate}
\item The textbox shall support basic operations including but not limited to
Copy, Paste, and Undo/Redo.
\item Thee textbox shall not implement word-wrap.	  
\end{enumerate}

\item[Paste from Clipboard button requirements] \hfill
\begin{enumerate}
\item This button shall clear the contents of the textbox and then place
the full contents of the clipboard into the textbox
\item If the clipboard does not contain text data, this button must be
disabled
\end{enumerate}

\item[Load from File button requirements] \hfill
\begin{enumerate}
\item Pressing this button will the standard Windows file open dialog.  It
shall be enabled for searching for \texttt{*.txt} and \texttt{*.zip} files.
\begin{enumerate}
\item  If a valid file is opened, the contents of the textbox shall be cleared
and then the full contents of the file shall be placed into the textbox.
Obviously, zipped files should be unzipped.
\item If the user presses cancel in the dialog, the contents of the textbox
shall not be changed
\end{enumerate}
\end{enumerate}

\item[\var{action} button requirements] \hfill
\begin{enumerate}
\item The contents of the textbox shall be scanned to determine if it contains
default script content (or empty script content), script content, or log
content.  If so, the button shall display ``Scan," ``Execute," or ``Analyze"
(respectively).
\item The button shall have differing behavior based on the inferred content of
the textbox:
\begin{enumerate}
\item If the button reads ``Scan," the tool shall proceed to the Running Screen
(section~\ref{sec:running_screen}) running the default script
(section~\ref{sec:default_script_sections}).
\item If the button reads ``Execute," the tool shall proceed to the Running
Screen (section~\ref{sec:running_screen}) running the supplied script.
\item If the button reads ``Analyze," the tool shall proceed to the Analysis
Screen (section~\ref{sec:analysis_screen}) displaying the parsed log.
\end{enumerate}
\item If the type of the content in the textbox cannot be inferred or the
content of the textbox is not syntactically valid, the button shall display
the text of the last inferred \var{ACTION}.  If the user presses the
\var{ACTION} button for an invalid script, the applicaiton must not continue. 
Instead, an error message must appear that informs the user that the script is
invalid and therefore the tool will not continue.
\end{enumerate}

\item[About button requirements] \hfill
\begin{enumerate}
\item This button shall display a screen that contains the license for this
project (section~\ref{sec:licensing}) as well as information for any other
tools used in this project.  This screen shall have a simple close button.
\item The behavior for this button is the same on all following windows. 
\end{enumerate}
\end{description}

\subsection{Running Screen} \label{sec:running_screen}
This screen will run a script.  For the purpose of these requirements, it is not
important whether the script is the default script or a custom script.  The
script shall automatically begin when this screen is displayed.  The Running 
Screen is presented in Figure~\ref{fig:gui_running}.

\begin{figure}[h]
  	\centering
	\includegraphics{figures/gui/Running.png}
  	\caption{GUI Running Screen}
  	\label{fig:gui_running}
\end{figure}

\begin{description}
\item[Textbox requirements] \hfill 
\begin{enumerate}
  \item The textbox shall be updated to display the (raw) log output of the
  currently running screen.  This textbox shall only be updated at a rate of 60
  Hz to avoid slowing down the GUI.
  \item The textbox must be scrollable.  It shall automatically scroll down to 
  follow the output by default.  If the user scrolls up for any reason, it shall
  no longer auto-scroll unless the user manually scrolls all the way down to the
  bottom of the output.
\end{enumerate}
\item[Progress bar requirements] \hfill 
\begin{enumerate}
  \item The progress bar shall contain the best estimate of the progress through
  the script.  This estimate can be something as simple as the completed script 
  actions divided by the total script actions.
\end{enumerate}
\item[Cancel button requirements] \hfill 
\begin{enumerate}
  \item The ``Cancel'' button must not be enabled if the script contains any 
  system-altering actions.  This can be determined by scanning the script in 
  advance.
  \item If the user presses the ``Cancel" button, the ``Next" button must  
  change to display ``Exit."  The text of the ``Cancel" button shall change to
  ``Re-run."  The behavior of both of these buttons should be self-explanatory.
  \item When the script completes, if the script was not a system-altering
  script, then the button should change to ``Re-run."  Otherwise, it shall
  remain displaying ``Cancel" and be disabled.
\end{enumerate}
\item[Window close button requirements] \hfill 
\begin{enumerate}
  \item The window close button must not be enabled if the script contains any 
  system-altering actions.  This can be determined by scanning the script in
  advance.
  \item If the user presses the window close button at any time that it is
  enabled, a Yes/No dialog should appear that reminds the user that the script
  output has not been saved yet.
\end{enumerate}
\item[Next button requirements] \hfill 
\begin{enumerate}
  \item The ``Next" button shall not be enabled until after the script 
  completes.
  \item When the user presses the ``Next" button, the tool shall proceed to the
  Run Completed Screen (section~\ref{sec:run_completed_screen}).
\end{enumerate}
\end{description}

\subsection{Run Completed Screen} \label{sec:run_completed_screen}
This screen enables a user to decide what to do with their script output (log
file).  This screen is presented in Figure~\ref{fig:gui_run_complete}.

\begin{figure}[h]
  	\centering
	\includegraphics{figures/gui/Run_Completed.png}
  	\caption{GUI Run Complete Screen}
  	\label{fig:gui_run_complete}
\end{figure}

\begin{description}
\item[Option requirements] \hfill
\begin{enumerate}
  \item By default, nothing shall be selected
  \item Both of the file Save fields shall default to the user's desktop
  (\verb|%userprofile%\Desktop\|) with the filenames \verb|Log.txt| and
  \verb|Log.zip|.
\end{enumerate}
\item[Next button requirements] \hfill
\begin{enumerate}
  \item The button must be disabled until at least one option is selected. It
  must return to being disabled if nothing is selected.
  \item The button shall have the following behavior when it is clicked:
  \begin{enumerate}
    \item If the options selected did not include ``Analyze output," then the
    tool shall proceed to the Finished Screen
    (section~\ref{sec:finished_screen}).
    \item If the options selected include ``Analyze output" and other options,
    then the other options shall execute and then the tool should proceed to
    the Analysis Screen (section~\ref{sec:analysis_screen}).
    \item If the only option selected was ``Analyze output," then the tool
    shall proceed to the Analysis Screen (section~\ref{sec:analysis_screen})
    after displaying a warning that the output will be otherwise unsavable.
  \end{enumerate}
\end{enumerate}
\item[Window close button requirements] \hfill
\begin{enumerate}
  \item If the user presses the window close button at any time that it is
  enabled, a Yes/No dialog shall appear that reminds the user that the script
  output has not been saved yet.
\end{enumerate}
\end{description}


\subsection{Analysis Screen} \label{sec:analysis_screen}
The analysis screen enables users to construct a script based on the output
from a log.  For the requirements listed in this section, it does not
matter what workflow the user used to get to this screen.  This screen is
presented in figure~\ref{fig:gui_analyze}.

\begin{figure}[h]
  	\centering
	\includegraphics{figures/gui/Analysis.png}
  	\caption{GUI Analysis Screen}
  	\label{fig:gui_analyze}
\end{figure}

\begin{description}
\item[Section heading requirements] \hfill
\begin{enumerate}
  \item Each separate section in the log shall be parsed into one of the gray
  section.  Sections in the log are described in section~\ref{sec:log_output}.
  \item Sections can be collapsed or expanded by the user.  The user can either
  press the entire section heading or the Chevron arrow to perform this action.
  By default, all sections will be expanded.
  \item The Chevron arrow shall point to the right for collapsed sections and
  downward for expanded sections.
  \item No animation is necessary for the collapse action or expand action.
  \item Hovering over a section shall slightly change the background color.
  \item Each corresponding line shall be listed under the section.  Some
  sections might not have any lines.  In this case, a single line shall appear
  with the following text in italics: ``No lines available for this section."
\end{enumerate}
\item[Line requirements] \hfill
\begin{enumerate}
  \item Each separate line of the log will be logged into a line underneath the
  corresponding section.
  \item Hovering over a line shall slightly change the background color.
\end{enumerate}
\item[Checkbox requirements] \hfill
\begin{enumerate}
  \item Depending on the line, there may be zero, one, or many actions available
  for the given line.  There shall be a checkbox to the left of the line for
  each action.  Checking a box indicates that the action shall be taken.
  \item A user shall be able to hover their mouse over any checkbox to
  determine what action the checkbox enables.
  \item In a given section, actions shall be grouped by actions.  Therefore,
  each column will only have one type of action in it.  If an action does not
  apply to a line, there shall simply be an empty slot where the checkbox would
  be.
  \item All checkboxes shall default to unchecked.
\end{enumerate}
\item[Aditional script actions requirements] \hfill
\begin{enumerate}
  \item The last section in the log must always be titled ``Additional script
  actions'' and will contain a textbox that will allow the user to specify
  additional script actions to take
  \item The textbox shall always have a minimum height of ten lines and will
  grow to always be one line longer than its contents
  \item The textbox shall not have its own scrollbars.  Rather, the scrollbars
  for the rest of the control are be sufficient
  \item The textbox must support basic operations including but not limited
  to Copy, Paste, and Undo/Redo.
  \item The textbox shall not implement word-wrap.
\end{enumerate}
\item[Next button requirements] \hfill
\begin{enumerate}
  \item The Next button shall display a Yes/No dialog warning the user that
  scripts are final and there is no going back.
\end{enumerate}
\end{description}

\subsection{Analysis Completed Screen}
This screen enables a user to decide what to do with the finished script.  This
screen is presented in Figure~\ref{fig:gui_analysis_completed}.

\begin{figure}[h]
  	\centering
	\includegraphics{figures/gui/Analysis_Completed.png}
  	\caption{GUI Analysis Completed Screen}
  	\label{fig:gui_analysis_completed}
\end{figure}

\begin{description}
\item[Option requirements] \hfill
\begin{enumerate}
  \item By default, nothing shall be selected
  \item Both of the file Save fields shall default to the user's desktop
  (\verb|%userprofile%\Desktop\|) with the filenames \verb|Script.txt| and
  \verb|Script.zip|.
\end{enumerate}
\item[Next button requirements] \hfill
\begin{enumerate}
  \item The button must be disabled until at least one option is selected. It
  must return to being disabled if nothing is selected.
  \item The button shall have the following behavior when it is clicked:
  \begin{enumerate}
    \item If the options selected did not include ``Run script," then the tool
    shall proceed to the Finished Screen (section~\ref{sec:finished_screen}).
    \item If the options selected include ``Run script" and other options, then
    the other options shall execute and then the tool shall proceed to the
    Running Screen (section~\ref{sec:running_screen}).
    \item If the only option selected was ``Run script," then the tool shall
    proceed to the Running Screen (section~\ref{sec:running_screen}).
  \end{enumerate}
\end{enumerate}
\item[Window close button requirements] \hfill
\begin{enumerate}
  \item If the user presses the window close button at any time that it is
  enabled, a Yes/No dialog must appear that reminds the user that the script
  has not been saved yet.
\end{enumerate}
\end{description}

\subsection{Finished Screen} \label{sec:finished_screen}
The finished screen provides the user with some confirmation that the actions
instructed of the tool were actually executed.  This is to prevent users from
becoming disoriented and thinking that the tool had crashed or something else
bad had happened.  This screen is presented in Figure~\ref{fig:gui_finished}.

\begin{figure}[h]
  	\centering
	\includegraphics{figures/gui/Finished.png}
  	\caption{GUI Finished Screen}
  	\label{fig:gui_finished}
\end{figure}

\begin{description}
\item[Text requirements] \hfill
\begin{enumerate}
  \item The text shall display some information about what happened.  The text
  should cover all possible combinations of copying material to the clipboard,
  saving one file, or saving two files.
\end{enumerate}
\item[Display file checkbox requirements] \hfill
\begin{enumerate}
  \item If a plaintext file was saved, then a checkbox shall be displayed that
  allows the user to specify if they want to view the file when the program
  exits.
  \item The default state of this checkbox shall be unchecked.
  \item Upon exiting, if this is checked, notepad shall be launched and display
  this file.
\end{enumerate}
\end{description}

\subsection{Flowchart}
Since there are several different branch points in the tool, the flow through
the tool is difficult to describe by simply using text.  The flow described in
the preceding sections is described in Figure~\ref{fig:gui_flowchart} in
flowchart form.

\begin{figure}[h!]
  	\centering
	\includegraphics{figures/gui/GUI_Flowchart.png}
  	\caption{GUI Flowchart}
  	\label{fig:gui_flowchart}
\end{figure}

Portions of this document are written in terms of the \verb|.REG| format
described by Regedit, a Widnows component. The syntax for the regedit script is
described in several locations, in particular  of \texttt{monospaced}
(\url{http://en.wikipedia.org/wiki/Windows_Registry#.REG_files}).

Additionally WOW64 (Windows on Windows 64) defines a system whereby registry
virtualization is in effect -- 32 bit programs are shown a virtualized view of
the 64 bit registry. Current versions of windows implement this by putting the
32 bit registry view in a key called ``\verb|Wow6432Node|'' in the root of each
hive in the registry. Additionally, Windows provides flags to the functions that open
registry keys, such as \verb|NtOpenKey| or \verb|RegCreateKeyEx|, which select
the correct view. Microsoft asks that applications be built in terms of the API
flags, because the specific name of \verb|Wow6432Node| is an implementation
detail which can change in future versions of Windows.

Therefore, we will define on top of the \verb|.REG| format a comment of
\verb|;32 Bit Registry View| or \verb|64 Bit Registry View|, which represent the
flag that would be passed into Windows when opening the indicated key.

\section{Log Output} \label{sec:log_output}
This section generally defines the form of Instalog's output. A log is split up
into several delimited portions called ``sections''. With the exception of
headers and footers, all sections begin with a line similar to the following
format:

\begin{verbatim}
================ ${Section Name} ===============
\end{verbatim}

That is, the name of the section with one space of padding, centered in a block
of equals (\verb|=|) signs 50 total characters wide. In the case that a tie
exists with respect to the centering, Instalog shall prefer placing the name of
the section farther to the right than to the left.

\subsection{Default Script Sections} \label{sec:default_script_sections}
Instalog's standard (that is, run without a script) output consists of the
following sections in the following order:
\begin{enumerate}
    \item Header
    \item Running Processes
    \item Machine PsuedoHJT Report
    \item $n$ User PseudoHJT Reports (One for each loaded user registry on the
    system)
    \item Mozilla Firefox (If Mozilla Firefox is installed)
    \item Google Chrome (If Google Chrome is installed)
    \item Created Last 30
    \item Find3M Report
    \item Event Viewer (If any relevant events need be reported)
    \item Machine Specifications
    \item Restore Points
    \item Installed Programs
    \item Footer
\end{enumerate}

\subsection{Additional Script Sections}
\noindent{}The following additional sections are available but they are not
generated in the default report:
\begin{description}
\item[DNS Check] Displayed if the \verb|:dnscheck| scripting section is used.
\item[Directory] Displayed if the \verb|:dirlook| scripting section is used.
\item[VirusTotal] Displayed if the \verb|:virustotal| scripting section is used.
\item[MRC Upload] Displayed if the \verb|:mrc| scripting section is used.
\item[Process Kill] Displayed if the \verb|:kill| scripting section is used.
\item[File Quarentine] Displayed if the \verb|:move| scripting section is used.
\item[Security Center] Displayed if the \verb|:securitycenter| scripting section
is used.
\item[Registry 32 Bit] Displayed if the \verb|:reg32| scripting action is used.
\item[Registry 64 Bit] Displayed if the \verb|:reg64| scripting action is used.
\end{description}

\subsection{Escaping Formats}
In order to meet the requirements of machine readability and human readability,
several escaping formats must be used depending on the type of data to produce
the most readable unambiguous representation given typical data collected from a
given location.

\subsubsection{General Escaping Format}
Generally, escaping MUST be done in a manner similar to most programming
languages, such as C, C++, Java, or similar, for quoted string escapes. Such an
escaping scheme is defined by three characters: an optional starting delimiter,
an optional termination delimiter, and an escape character. For instance, in C,
the starting delimiter is the quote mark, \verb|"|, the ending delimiter is also
a quote mark \verb|"|, and the escape character is the backslash \verb|\|.

For most types of information Instalog enumerates, such as Windows file paths,
the exact method C uses is unsuitable; backslashes occur far too often inside
file paths for the backslash as an escape character to make a good choice.
Legacy applications like the command processor (\texttt{cmd.exe}) get around
such problems by implementing complicated escaping schemes, but these are
designed with specific input data in mind (file paths and command line
switches, for instance), which is not the case for most reported data.

\label{generalescape}
Therefore, we define a general escape function that works in a manner similar to
C's string literal escapes, but which allows arbitrary starting, termination,
and escaping characters. The input to the escape function is raw data obtained
from some data source, and the output is the same data with the following
textual replacements:

\begin{itemize}
    \item \var{EscapeCharacter} is replaced with
    \var{EscapeCharacter}\var{EscapeCharacter}.
    \item \var{RightDelimiter}, if defined,  is replaced with
    \var{EscapeCharacter}\var{RightDelimiter}.
    \item The null character (ASCII \verb|0x00|) is replaced with
    \var{EscapeCharacter}\texttt{0}.
    \item The backspace character (ASCII \verb|0x08|) is replaced with
    \var{EscapeCharacter}\texttt{b}.
    \item The form feed character (ASCII \verb|0x0C|) is replaced with
    \var{EscapeCharacter}\texttt{f}.
    \item The newline character (ASCII \verb|0x0A|) is replaced with
    \var{EscapeCharacter}\texttt{n}.
    \item The carriage return character (ASCII \verb|0x0D|) is replaced with
    \var{EscapeCharacter}\texttt{r}.
    \item The horizontal tab character (ASCII \verb|0x09|) is replaced with
    \var{EscapeCharacter}\texttt{t}.
    \item The vertical tab character (ASCII \verb|0x0B|) is replaced with
    \var{EscapeCharacter}\texttt{v}.
    \item ASCII characters which are not in the above list but are unprintable
    (that is, ASCII \verb|0x00| - \verb|0x1F|; \verb|0x7F|) are
    replaced with \var{EscapeCharacter}\texttt{xHH}, where \texttt{HH} is the
    hexadecimal representation of the numeric value of the given character.
    \item Non-ASCII characters (\verb|U+0080| and above) are replaced with
    \var{EscapeCharacter}\texttt{uHHHH}, where \texttt{HHHH} is the hexadecimal
    representation of the numeric value of the character.
    \item Unicode characters requiring a surrogate pair in UTF-16 (that is,
    greater than U+FFFF) are represented as two normal Unicode
    (\var{EscapeCharacter}\texttt{uHHHH}) escapes matching the UTF-16 surrogate
    pair. (This is exactly how they'd be represented in Windows itself.)
    \item Forum software destroys significant whitespace; but escaping every
    space would be impractical. Where there is more than a single consecutive
    space, the second and later spaces are escaped. (That is, replaced with
    ``\var{EscapeCharacter}\texttt{ }''.)
    \item Any instance of \var{EscapeCharacter} followed by an unused character
    in the above list is equivalent to that character with no escape. This will
    never be generated by the escaping functionality in Instalog for general
    escaping, but is valid for the reverse, unescaping, operation.
\end{itemize}

\subsubsection{URL Escaping Format} \label{urlescape}
URL escaping matches the general escaping format above, except has the
additional constraint of forum software which tries to convert URLs into links.
It is desirable to inhibit this behavior of the forum software, so that board
owners need not worry about linking to malicious websites. Therefore, URL
escaping is defined to be general escaping with the additional requirement that
when the string ``\texttt{http}'' (case insensitive) appears in the source, it
is replaced with ``\texttt{htt}\var{EscapeCharacter}\texttt{p}''. This inhibits
forum software's URL behavior which looks for the protocol prefix in order to
determine what portions of a post indicate a link. (Note that due to the last
requirement of general escaping this will cause no effect on the unescaped data).

\subsection{Header}
The header of an Instalog report consists of 4 lines, which are always displayed
and shown in the same order as defined in this document. No component of the
header is associated with any type of fix information.

\subsubsection{Line One}
The first line lists information about Instalog itself. It is of the form:
\begin{verbatim}
Instalog ${Version}${SafebootState}
\end{verbatim}

\var{Version} is the version number of the current release of Instalog. This
allows remote determination of cases where a user needs to upgrade their current
copy before malware removal or system repair can continue safely.

\var{SafebootState} is the string ``\verb| MINIMAL|'' if the system is currently
booted into Windows' \textit{Safe Mode}, or the string ``\verb| NETWORK|'' if
the systemis booted into \textit{Safe Mode with Networking}, or nothing if the
machine was booted normally.

\subsubsection{Line Two}
The second line lists information about the user running Instalog, and the local
date and time of the system when the log was generated. It has the form:
\begin{verbatim}
Run by ${UserName} at ${Y}-${M}-${D} ${H}:${M}:${S}.${Milli} [GMT ${TimeZone}]
\end{verbatim}

\var{UserName} is the current display name of the logged in user, escaped with
the general escaping format defined in \ref{generalescape}, using a left
delimiter of a double quote (\verb|"|), a right delimiter of a double
quote (\verb|"|), and an escape character backslash (\verb|\|).

\var{Y}, \var{M}, \var{D}, \var{H}, \var{M}, \var{S}, and \var{Milli} are
replaced with numeric representations of the current local date and time (that
is, year, month, day, hour, minute, second, and millisecond, respectively).

\var{TimeZone} is a one sign, three digit, and one decimal point representation
of the time zone of the machine taking the report. For instance, Eastern
Standard Time is \verb|-4.00| or \verb|-5.00|, while Moscow Standard Time would
be \verb|+4.00|. (The extra two digits are to allow for locales with half and
quarter hour time zones)

\subsubsection{Line Three}
The third line is designed to indicate when important exploitable applications
need to be updated, by listing their versions. It has the form:
\begin{verbatim}
IE: ${IE} Java: ${Java} Flash: ${Flash} Adobe: ${AdobeReader}
\end{verbatim}

The variables are the installed versions of Microsoft's Internet Explorer,
Oracle's Java, Adobe's Flash, and Adobe's Adobe Reader. In the event one or more
of these applications are not installed then their version is listed as
``\verb|None|''.

\subsubsection{Line Four}
The fourth line contains information about the current Windows installation and
memory state. It is of the form:
\begin{verbatim}
Microsoft Windows ${WindowsVersion} ${WindowsEdition} ${ProcessorArchitecture}
${Major}.${Minor}.${Build}.${ServicePack} ${FreeRam}/${TotalRam} MB Free
\end{verbatim}

Newlines in the above are a consequence of this document and are not present in
the output.

\var{WindowsVersion} is the ``string name'' of the version of Windows in use,
such as ``XP'', ``Vista'', or ``7''. \var{WindowsEdition} is the edition of the
same; such as ``Home'', ``Professional'', or ``Ultimate''. The values
\var{Major}, \var{Minor}, \var{Build}, and \var{ServicePack} match the current
version information of the operating system in use.

\var{ProcesserArchitecture} is either the string ``\verb|x86|'' or
``\verb|x64|'', matching the installed operating system type. (Note that this is
\textit{not} the capability of the current processor)

Finally, \var{FreeRam} and \var{TotalRam} are the number of Mebibytes of memory
that are available for use by programs on the current running system.

Examples:
\begin{verbatim}
Microsoft Windows Vista Professional N x86 6.0.6000.0 1023/8096 MB Free
Microsoft Windows 7 Ulimate x64 6.1.7601.1 4547/8071 MB Free
\end{verbatim}

\subsection{PseudoHJT Machine Report}
The information in the PseudoHJT report MUST closely match the information
displayed by the original HJT tool. The format MUST closely match the syntax of
a similar predecessor, DDS, so that forum volunteers need not learn
significantly different syntax to that which they know. This format MUST be
sparse and human readable. Perhaps most importantly, despite matching the
original HJT in terms of information conveyed, the format itself MUST be
different enough to avoid legal action by TrendMicro against Instalog's authors.

Generally speaking, the PseudoHJT report attempts to list all relevant loading
points on a Windows machine, as well as user settings for the Internet Explorer
browser; such as home page, search provider, and title settings. The report is
heavily whitelisted. Items which are defaults on Windows MUST NOT be emitted,
unless whitelisting has been disabled. Specific whitelisting schemes are given
per type of line shown in the PseudoHJT report.

Each line in the PseudoHJT report is given a unique prefix, which is not shared
by other line types. This allows the line to be unambiguously understood by the
Instalog GUI, and other kinds of inspection tools.

\subsubsection{Default Page URL}
\begin{description}
\item[Rationale] \hfill \\
The User Default Page URL is the location(s) of pages opened when a user creates
a new Microsoft Internet Explorer window, or starts the browser for the first
time. This value may be overridden by a user-specific setting. This setting
affects 32 bit versions of Internet Explorer only.
\item[Data Sources] \hfill
\vspace{-\baselineskip}
\begin{verbatim}
; 32 Bit Registry View
[HKEY_LOCAL_MACHINE\Software\Microsoft\Internet Explorer]
"Default_Page_URL"="${url}"
\end{verbatim}
\item[Log Format] \hfill
\vspace{-\baselineskip}
\begin{verbatim} 
DefaultPageUrl=${url}
\end{verbatim}
\item[Output Description] \hfill \\
The variable \var{url} is escaped using the URL escaping scheme defined in
\ref{urlescape}, where the escape character is the hash mark (\verb|#|). 
\item[Whitelist Considerations] \hfill \\
The default value for this entry
differs depending on the version of Internet Explorer and Windows currently in
use. Testing will need to be undertaken against supported operating systems in
order to determine which values to hide.
\item[Fix Considerations] \hfill \\
Valid options: fix registry. Generated
script:
\vspace{-\baselineskip}
\begin{verbatim}
:reg32
[HKEY_LOCAL_MACHINE\Software\Microsoft\Internet Explorer]
"Default_Page_URL"="about:blank"
\end{verbatim}
\end{description}

\subsubsection{64 Bit Default Page URL}
\begin{description}
\item[Rationale] \hfill \\
The User Default Page URL is the location(s) of pages opened when a user creates
a new Microsoft Internet Explorer window, or starts the browser for the first
time. This value may be overridden by a user-specific setting. This setting
affects 64 bit versions of Internet Explorer only.
\item[Data Sources] \hfill
\vspace{-\baselineskip}
\begin{verbatim}
; 64 Bit Registry View
[HKEY_LOCAL_MACHINE\Software\Microsoft\Internet Explorer]
"Default_Page_URL"="${url}"
\end{verbatim}
\item[Log Format] \hfill
\vspace{-\baselineskip}
\begin{verbatim} 
DefaultPageUrl64=${url}
\end{verbatim}
\item[Output Description] \hfill \\
The variable \var{url} is escaped using the URL escaping scheme defined in
\ref{urlescape}, where the escape character is the hash mark (\verb|#|). 
\item[Whitelist Considerations] \hfill \\
The default value for this entry
differs depending on the version of Internet Explorer and Windows currently in
use. Testing will need to be undertaken against supported operating systems in
order to determine which values to hide.
\item[Fix Considerations] \hfill \\
Valid options: fix registry. Generated
script:
\vspace{-\baselineskip}
\begin{verbatim}
:reg64
[HKEY_LOCAL_MACHINE\Software\Microsoft\Internet Explorer]
"Default_Page_URL"="about:blank"
\end{verbatim}
\end{description}

\subsection{PseudoHJT User Report}

\subsubsection{Default Page URL}
\begin{description}
\item[Rationale] \hfill \\
The User Default Page URL is the location(s) of pages opened when a user creates
a new Microsoft Internet Explorer window, or starts the browser for the first
time. If this value is set, it overrides the machine-wide Default Page URL
value. This setting affects 32 bit versions of Internet Explorer only.
\item[Data Sources] \hfill
\vspace{-\baselineskip}
\begin{verbatim}
; 32 Bit Registry View
[HKEY_CURRENT_USER\Software\Microsoft\Internet Explorer]
"Default_Page_URL"="${url}"
\end{verbatim}
\item[Log Format] \hfill
\vspace{-\baselineskip}
\begin{verbatim} 
DefaultPageUrl=${url}
\end{verbatim}
\item[Output Description] \hfill \\
The variable \var{url} is escaped using the URL escaping scheme defined in
\ref{urlescape}, where the escape character is the hash mark (\verb|#|). 
\item[Whitelist Considerations] \hfill \\
The default value for this entry
differs depending on the version of Internet Explorer and Windows currently in
use. Testing will need to be undertaken against supported operating systems in
order to determine which values to hide.
\item[Fix Considerations] \hfill \\
Valid options: fix registry. Generated
script:
\vspace{-\baselineskip}
\begin{verbatim}
:reg32
[HKEY_CURRENT_USER\Software\Microsoft\Internet Explorer]
"Default_Page_URL"="about:blank"
\end{verbatim}
\end{description}

\subsubsection{64 Bit Default Page URL}
\begin{description}
\item[Rationale] \hfill \\
The User Default Page URL is the location(s) of pages opened when a user creates
a new Microsoft Internet Explorer window, or starts the browser for the first
time. If this value is set, it overrides the machine-wide Default Page URL
value. This setting affects 64 bit versions of Internet Explorer only.
\item[Data Sources] \hfill
\vspace{-\baselineskip}
\begin{verbatim}
; 64 Bit Registry View
[HKEY_CURRENT_USER\Software\Microsoft\Internet Explorer]
"Default_Page_URL"="${url}"
\end{verbatim}
\item[Log Format] \hfill
\vspace{-\baselineskip}
\begin{verbatim} 
DefaultPageUrl=${url}
\end{verbatim}
\item[Output Description] \hfill \\
The variable \var{url} is escaped using the URL escaping scheme defined in
\ref{urlescape}, where the escape character is the hash mark (\verb|#|). 
\item[Whitelist Considerations] \hfill \\
The default value for this entry
differs depending on the version of Internet Explorer and Windows currently in
use. Testing will need to be undertaken against supported operating systems in
order to determine which values to hide.
\item[Fix Considerations] \hfill \\
Valid options: fix registry. Generated
script:
\vspace{-\baselineskip}
\begin{verbatim}
:reg64
[HKEY_CURRENT_USER\Software\Microsoft\Internet Explorer]
"Default_Page_URL"="about:blank"
\end{verbatim}
\end{description}


\subsection{Header}
The header of an Instalog report consists of 4 lines, which are always displayed
and shown in the same order as defined in this document. No component of the
header is associated with any type of fix information.

\subsubsection{Line One}
The first line lists information about Instalog itself. It is of the form:
\begin{verbatim}
Instalog ${Version}${SafebootState}
\end{verbatim}

\var{Version} is the version number of the current release of Instalog. This
allows remote determination of cases where a user needs to upgrade their current
copy before malware removal or system repair can continue safely.

\var{SafebootState} is the string ``\verb| MINIMAL|'' if the system is currently
booted into Windows' \textit{Safe Mode}, or the string ``\verb| NETWORK|'' if
the systemis booted into \textit{Safe Mode with Networking}, or nothing if the
machine was booted normally.

\subsubsection{Line Two}
The second line lists information about the user running Instalog, and the local
date and time of the system when the log was generated. It has the form:
\begin{verbatim}
Run by ${UserName} at ${Y}-${M}-${D} ${H}:${M}:${S}.${Milli} [GMT ${TimeZone}]
\end{verbatim}

\var{UserName} is the current display name of the logged in user, escaped with
the general escaping format defined in \ref{generalescape}, using a left
delimiter of a double quote (\verb|"|), a right delimiter of a double
quote (\verb|"|), and an escape character backslash (\verb|\|).

\var{Y}, \var{M}, \var{D}, \var{H}, \var{M}, \var{S}, and \var{Milli} are
replaced with numeric representations of the current local date and time (that
is, year, month, day, hour, minute, second, and millisecond, respectively).

\var{TimeZone} is a one sign, three digit, and one decimal point representation
of the time zone of the machine taking the report. For instance, Eastern
Standard Time is \verb|-4.00| or \verb|-5.00|, while Moscow Standard Time would
be \verb|+4.00|. (The extra two digits are to allow for locales with half and
quarter hour time zones)

\subsubsection{Line Three}
The third line is designed to indicate when important exploitable applications
need to be updated, by listing their versions. It has the form:
\begin{verbatim}
IE: ${IE} Java: ${Java} Flash: ${Flash} Adobe: ${AdobeReader}
\end{verbatim}

The variables are the installed versions of Microsoft's Internet Explorer,
Oracle's Java, Adobe's Flash, and Adobe's Adobe Reader. In the event one or more
of these applications are not installed then their version is listed as
``\verb|None|''.

\subsubsection{Line Four}
The fourth line contains information about the current Windows installation and
boot state. It is of the form:
\begin{verbatim}
Microsoft Windows ${WindowsVersion} ${WindowsEdition} ${ProcessorArchitecture}
${Major}.${Minor}.${Build}.${ServicePack} ${FreeRam}/${TotalRam} MB Free
\end{verbatim}

Newlines in the above are a consequence of this document and are not present in
the output.

\var{ProcesserArchitecture} is either the string ``\verb|x86|'' or
``\verb|x64|'', matching the installed operating system type. (Note that this is
\textit{not} the capability of the current processor)

Examples:
\begin{verbatim}
Microsoft Windows Vista Professional N x86 6.0.6000.0 1023/8096 MB Free
Microsoft Windows 7 Ulimate x64 6.1.7601.1 4547/8071 MB Free
\end{verbatim}

\subsection{Running Processes}

\subsection{PseudoHJT Machine Report}
The information in the PseudoHJT report MUST closely match the information
\end{document}
