%%This is a very basic article template.
%%There is just one section and two subsections.
\documentclass[letterpaper,12pt]{article}
\usepackage[margin=1in]{geometry}
\usepackage{url}

\title{Instalog Requirements Specification}
\author{
Billy R. O'Neal III \\
Jacob Snyder
}

\begin{document}

\maketitle

\section{Introduction}
Instalog is a senior project by Jacob Snyder (jrs213@case.edu) and Billy O'Neal
(bro4@case.edu), which is designed to gather information from Microsoft Windows
installations, for the purpose of malware removal and system repair.  It must
generate a human and machine readable report which assists end users, remote
experts, and local administrators with issue diagnosis and malware removal.

Instalog is inspired by several similar tools which all share some basic
functionality.  In many ways, Instalog can be viewed as an evolution of
these tools:
\begin{itemize}
    \item TrendMicro's {\em Hijack This} (HJT)
    \item ``sUBs'' {\em Doesn't Do Squat} (DDS)
    \item ``random/random'''s {\em Random's System Information Tool} (RSIT)
    \item ``OldTimer'''s {\em OTA}, {\em OTS}, and {\em OTL} (formerly
    OTAnalyzeIt, OTScanIt, and OTListIt, respectively)
    \item Sysinternals' {\em Autoruns}
    \item Runscanner's {\em Runscanner}
\end{itemize}
all of which purport to accomplish similar goals to Instalog. However, each of
these tools has bugs or specific behavior which cause problems for at least one
of Instalog's three intended user groups.

Specifically, the above tools contain one or more of the above problems:
\begin{itemize}
    \item Incorrect handling and escaping of log data
    \item Lack of published specifications, documentation, or source code
    \item Outstanding bugs that the authors are unwilling or unable to fix
    \item Lack of scriptability, for the purposes of modifying log output and
    malware removal.
    \item Lack of 64 bit support.
    \item Lack of Unicode support.
    \item Lack of enumeration of some types of useful log information.
\end{itemize}
Instalog will attempt to solve those problems by combining characteristics of
the above tools which are deemed useful, while mixing in a few tricks of it's
own.

\subsection{Document Conventions}
Within the scope of this document, computer output or other information that is
to be taken literally is written in {\tt fixed width text}. In some instances, a
block of fixed width text will be surrouned by non-fixed width quotes ``{\tt\
like this}''. In such cases, the quotes are not significant, and there will
there will (typically) be leading or trailing space around the fixed width
block, which is significant and MUST NOT be removed. Variables, which are
replaced with some content, are written of the form {\tt \$\{name\}}, and will
be explained in greater detail in prose surrounding a given block of {\tt fixed
width} text.

\subsection{Intended Audience}
Instalog is designed with three types of target users in mind. These ``user
classes'' are listed in the following sections.

\subsubsection{Home Users}
For a typical home user, Instalog MUST NOT display a complicated interface, and
must make it relatively difficult to misstep and take a wrong action. Few
options need be presented, such as the ability to generate a default report and
the ability to take a given script and run it on a target machine. Complicated
features such as analysis MUST NOT be displayed; though they may appear as
options that are, by default, deselected.

\subsubsection{Administrators}
Administrators are similar to home users in that they are physically working at
a computer being examined, but they are different in that they have the inten of
repairing their own computer or the computer of a client. They wish to see
analysis features and more possible options. Instalog MUST provide a means for
Administrators to use it's analysis features without manual saving and reloading
of log files.

\subsubsection{Forum Experts}
Forum Experts help typical end users repair their machines remotely over
self-help forums such as \url{bleepingcomputer.com} or \url{geekstogo.com}.
These usersremotely, and likely will never see a given target machine.
Instalog MUST produce log formats that are human readable in the vast majority
of cases, but which can be passed through common forum software such as Invision
Power Board, phpBB, or vBulletin without destruction of information.
Unforunately, this makes common data exchange formats such as JSON and XML
unsuitable. 

Moreover, as obtaining additional information from a machine may
have lead times of several days, Instalog's report must be unambiguous; that is,
no two possible system configurations may produce the same output. Experts can
also benefit from log analysis features. Finally, Experts need to be able to
write simple, human readable scripts to perform actions to fix a user's machine
remotely.

\section{Grapherical User Interface}
\section{Log Output}
\subsection{PseudoHJT Report}
The information in the PseudoHJT report MUST closely match the information
displayed by the original HJT tool. The format MUST closely match the syntax of
a similar predecessor, DDS, so that forum voulenteers need not learn
significantly different syntax to that which they know. This format MUST be
sparse and human readable. Perhaps most importantly, despite matching the
original HJT in terms of information conveyed, the format itself MUST be
different enough to avoid legal action by TrendMicro against Instalog's authors.

Generally speaking, the PseudoHJT report attempts to list all relevant loading
points on a Windows machine, as well as user settings for the Internet Explorer
browser; such as home page, search provider, and title settings. The report is
heavily whitelisted. Items which are defaults on Windows MUST NOT be emitted,
unless whitelisting has been disabled. Specific whitelisting schemes are given
per type of line shown in the PseudoHJT report.

Each line in the PseudoHJT report is given a unique prefix, which is not shared
by other line types. This allows the line to be unambiguously understood by the
Instalog GUI, and other kinds of inspection tools.

\end{document}
