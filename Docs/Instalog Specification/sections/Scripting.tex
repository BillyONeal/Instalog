\section{Scripting}
This section defines the scripting language that Instalog accepts for building
scripts.  The actions defined in this section will enable the tool to fix a wide
variety of problems that may be revealed by the system scan.  This being said,
the scripting language is not designed to be a master fix tool.  Rather, it is
designed to fix a wide majority of problems.  However, in some cases such as
uninstalling an application, other actions outside of Instalog's actions may be
required. 

\subsection{Running a Script}
Instalog shall go through the following steps when running a script.  
\subsubsection{Validation}
When Instalog is presented with a script, the first thing that it should do is
attempt to parse the script to ensure it does not have any syntax errors.  If
any syntax errors are present, Instalog \textbf{must not} proceed with executing
the script and shall instead return an error describing which line of the script
the syntax error was found on.
\subsubsection{Order Optimization}
Since scripts are written by users, they might not be presented in an order that
makes sense or is efficient.  For example, it does not make any sense to process
a script that quarantines files and then kills a process.  Rather, the order
should be optimized so that script actions have the greatest order of success.

Before reordering the script, script actions must be merged.  For example, if a
script defines two separate kill process sections, they shall be merged into one
kill process section.  Then, the script shall be rearranged into the following
order:
\begin{enumerate}
%TODO: Determine order
  \item 
\end{enumerate}
\subsubsection{Backup Folder Creation} \label{sec:backup_folder}
%TODO: ``system altering'' needs to be defined
If a script is determined to be system-altering, a folder must be created to
save backups and any intermediate output files.  Each script run shall create a
subfolder within \verb|C:\Instalog\| that is named according to the standard
date format defined in \ref{stddate}.  A copy of the script shall be saved in
this folder with the name ``\verb|Script.txt|''.  In addition, the script's output
shall be saved in this folder with the name ``\verb|Output.txt|''.  After each
different script action completes, this file must be written to.  This is done
so that if the system crashes while a script is executing, it will be possible
to determine where in the script execution it crashed and therefore which
actions successfully completed.  
\subsubsection{Execution}
Each action in the script shall be executed in the optimized order.  After each
action completes, the intermediate log shall be written to as explained in
\label{sec:backup_folder}.

\subsection{General Syntax}
While Instalog scripts are designed to be constructed by the GUI, the scripting
language is designed in such a way that editing them by hand with a simple text
editor is possible.  

Actions are listed in scripts in different action sections.  Scripts are built
by combining 1 or more action sections together.  Each action section is defined
as:
\begin{verbatim}
:${action} ${arg}
${items}
\end{verbatim}
\var{action} indicates which script action will be taken.  Some script actions
accept an optional argument, \var{arg}.  If the action takes an argument,
then \var{arg} will be whatever string follows the one space character after
\var{action}.  For example, if there were three spaces following \var{action}
and then some other characters, \var{arg} would contain two spaces and then
those characters.  Most actions expect a list of items to perform an action on. 
This is provided as \var{list}, which is a collection of 0 or more lines of
\var{item}'s for the script action to operate upon.  Each \var{action} and
\var{item} in \var{items} must be on separate lines in the script.

\subsection{Output Format}
The output format is defined differently for actions that are system-altering
and not system-altering.
\subsubsection{System-altering Action Output Format}
Actions that modify a system must either succeed or fail after they have been
performed.  If the action succeeded, it shall be emitted to the log as
\begin{verbatim}
[ OK ] ${message}
\end{verbatim}
where \var{message} is some information about the action.  Similarly, if the
action failed, it shall be emitted to the log as
\begin{verbatim}
[FAIL] ${message}
\end{verbatim}
\subsubsection{Non-system-altering Action Output Format}
Actions that are not system-altering can simply have their status message
emitted to the log.

\subsection{Backup Actions}
An important aspect of Instalog's scripting actions is that most are designed to
be ``safe,'' meaning that changes that are made are designed to be backed up in
some manner so that if something goes wrong, the change can be reverted. 
Several backup procedures are described in this section to provide this.  While
these procedures are in place, this tool will not provide functionality to
restore the backups it takes.  Rather, this will be left up to the user and
other external tools.

Instalog shall use two different backup procedures.  Each script action must
dictate which backup procedure it uses.  
\subsubsection{File Backup}
Rather than deleting files, Instalog shall ``quarantine'' them, that is,
Instalog will move them from their current location to some safe location. 
Quarantined files will be placed into a zip container in the corresponding
backup folder for the current script.  This will allow quarantined files to be
easy to associate with a given script run and therefore easy to restore if the
need arises.  

All files quarantined by a script shall be placed in a zip file named
\verb|Files.zip|, which must be located in the backup directory defined in
\ref{sec:backup_folder}.  Files shall be stored in the zip file with the same
path that they have on disk with the exception that the drive letter shall be
the top-most directory.  For example, the file \verb|C:\dir\sub\file.txt| shall
be saved in the zip as \verb|C\dir\sub\file.txt|.
\subsubsection{Registry Backup} 
If a script will modify the registry in any way, the registry must first be
backed up in its entirety.  The preferred method for registry backups is a
restore point.  Therefore, Instalog shall first attempt to create a restore
point.  If the restore point creation is successful, then this is a sufficient
backup.  However, if the restore point creation fails for some reason, then the
entire registry hive must be dumped manually.  Each registry file shall be saved
in the backup defined in \ref{sec:backup_folder}.

\subsection{Default Script Sections} \label{sec:default_script_sections}
Instalog's standard (that is, run without a script) output consists of the
following sections in the following order:
\begin{enumerate}
    \item Header
    \item Running Processes
    \item Machine PsuedoHJT Report
    \item $n$ User PseudoHJT Reports (One for each loaded user registry on the
    system)
    \item Mozilla Firefox (If Mozilla Firefox is installed)
    \item Google Chrome (If Google Chrome is installed)
    \item Created Last 30
    \item Find3M Report
    \item Event Viewer (If any relevant events need be reported)
    \item Machine Specifications
    \item Restore Points
    \item Installed Programs
    \item Footer
\end{enumerate}

\subsection{Additional Script Sections}
\noindent{}The following additional sections are available but they are not
generated in the default report:
\begin{description}
\item[DNS Check] Displayed if the \verb|:dnscheck| scripting section is used.
\item[Directory] Displayed if the \verb|:dirlook| scripting section is used.
\item[VirusTotal] Displayed if the \verb|:virustotal| scripting section is used.
\item[MRC Upload] Displayed if the \verb|:mrc| scripting section is used.
\item[Process Kill] Displayed if the \verb|:kill| scripting section is used.
\item[File Quarantine] Displayed if the \verb|:move| scripting section is used.
\item[Security Center] Displayed if the \verb|:securitycenter| scripting section
is used.
\item[Registry] Displayed if the \verb|:registry| scripting action is used.
\end{description}

\subsection{Script Actions}
\subsubsection{Registry Backup}
\subsubsection{Kill Process}
\subsubsection{VirusTotal Upload}
\subsubsection{Bleeping Computer Upload}
\subsubsection{Move (File Quarantine)}
\subsubsection{Hosts File Reset}
\subsubsection{Mozilla Firefox}
\subsubsection{Google Chrome}
\subsubsection{Security Center}
\subsubsection{Registry Action}
%32/64 bit
%Remove single MultiSZ value
%ExpandSZ as a string
%MultiSZ as a string
%Remove comma seperated bit like appinitdlls
\subsubsection{LSP Chain}
\subsubsection{Scan Actions}
\subsubsection{Directory Listing}
\subsubsection{DNS Check}