\section{Scripting}
This section defines the scripting language that Instalog accepts for building
scripts.  The actions defined in this section will enable the tool to fix a wide
variety of problems that may be revealed by the system scan.  This being said,
the scripting language is not designed to be a master fix tool.  Rather, it is
designed to fix a wide majority of problems.  However, in some cases such as
uninstalling an application, other actions outside of Instalog's actions may be
required. 

\subsection{Running a Script}
Instalog shall go through the following steps when running a script.  
\subsubsection{Validation}
When Instalog is presented with a script, the first thing that it should do is
attempt to parse the script to ensure it does not have any syntax errors.  If
any syntax errors are present, Instalog \textbf{must not} proceed with executing
the script and shall instead return an error describing which line of the script
the syntax error was found on.
\subsubsection{Reordering}
Since scripts are written by users, they might not be presented in an order that
makes sense or is efficient.  For example, it does not make any sense to
perform scanning actions before repair actions.  Therefore, the order should be
optimized so that script actions have the greatest order of success.

Before reordering the script, script actions must be merged.  For example, if a
script defines two separate kill process sections, they shall be merged into one
kill process section.  Then, the script shall be rearranged into the following
order:
\begin{enumerate}
%TODO: Determine order
  \item 
\end{enumerate}
\subsubsection{Backup Folder Creation} \label{sec:backup_folder}
If a script contains any repair actions defined in \ref{sec:repair_actions}, a
folder must be created to save backups and any intermediate output files.  Each
script run shall create a subfolder within \verb|C:\Instalog\| (if this folder
doesn't exist, it shall be created) that is named according to the standard date
format defined in \ref{stddate}.  A copy of the script shall be saved in this
folder with the name ``\verb|Script.txt|''.  In addition, the script's output
shall be saved in this folder with the name ``\verb|Output.txt|''.  After each
different script action completes, this file must be written to.  This is done
so that if the system crashes while a script is executing, it will be possible
to determine where in the script execution it crashed and therefore which
actions successfully completed.
\subsubsection{Execution}
Each action in the script shall be executed in the optimized order.  After each
action completes, the intermediate log shall be written to as explained in
\label{sec:backup_folder}.

\subsection{Script Syntax} \label{sec:script_syntax}
While Instalog scripts are designed to be constructed by the GUI, the scripting
language is designed in such a way that editing them by hand with a simple text
editor is possible.  

Actions are listed in scripts in different action sections.  Scripts are built
by combining one or more action sections together.  Each action section is
defined as:
\begin{verbatim}
:${action} ${arg}
${items}
\end{verbatim}
\var{action} indicates which script action will be taken.  Some script actions
accept an optional argument, \var{arg}.  If the action takes an argument,
then \var{arg} will be whatever string follows the one space character after
\var{action}.  For example, if there were three spaces following \var{action}
and then some other characters, \var{arg} would contain two spaces and then
those characters.  Most actions expect a list of items to perform an action on. 
This is provided as \var{list}, which is a collection of zero or more lines of
\var{item}'s for the script action to operate upon.  Each \var{action} and
\var{item} in \var{items} must be on separate lines in the script.

\subsection{Output Format} \label{sec:section_separator}
The output log can be seen as a combination of three elements: a header
containing general information about the system, information from the script
separated into separate sections, and a footer designating the end of the log. 
Each of these separate elements is described below.

\subsubsection{Header}
The header of an Instalog report consists of 4 lines, which are always displayed
and shown in the same order as defined in this document. No component of the
header is associated with any type of fix information.

The first line lists information about Instalog itself. It is of the form:
\begin{verbatim}
Instalog ${Version}${SafebootState}
\end{verbatim}

\var{Version} is the version number of the current release of Instalog. This
allows remote determination of cases where a user needs to upgrade their current
copy before malware removal or system repair can continue safely.

\var{SafebootState} is the string ``\verb| MINIMAL|'' if the system is currently
booted into Windows' \textit{Safe Mode}, or the string ``\verb| NETWORK|'' if
the system is booted into \textit{Safe Mode with Networking}, or nothing if the
machine was booted normally.

The second line lists information about the user running Instalog, and the local
date and time of the system when the log was generated. It has the form:
\begin{verbatim}
Run by ${UserName} at ${Date} [GMT ${TimeZone}]
\end{verbatim}

\var{UserName} is the current display name of the logged in user, escaped with
the general escaping format defined in \ref{generalescape}, using a left
delimiter of a double quote (\verb|"|), a right delimiter of a double
quote (\verb|"|), and an escape character backslash (\verb|\|).

\var{Date} is the current date when the log report was started being generated,
using the millisecond date format specified in \ref{millidate}.

\var{TimeZone} is a one sign, three digit, and one decimal point representation
of the time zone of the machine taking the report. For instance, Eastern
Standard Time is \verb|-4.00| or \verb|-5.00|, while Moscow Standard Time would
be \verb|+4.00|. (The extra two digits are to allow for locales with half and
quarter hour time zones)

The third line is designed to indicate when important exploitable applications
need to be updated, by listing their versions. It has the form:
\begin{verbatim}
IE: ${IE} Java: ${Java} Flash: ${Flash} Adobe: ${AdobeReader}
\end{verbatim}

The variables are the installed versions of Microsoft's Internet Explorer,
Oracle's Java, Adobe's Flash, and Adobe's Adobe Reader. In the event one or more
of these applications are not installed then their version is listed as
``\verb|None|''.

The fourth line contains information about the current Windows installation and
memory state. It is of the form:
\begin{verbatim}
Microsoft Windows ${WindowsVersion} ${WindowsEdition} ${ProcessorArchitecture}
${Major}.${Minor}.${Build}.${ServicePack} ${FreeRam}/${TotalRam} MB Free
\end{verbatim}

Newlines in the above are a consequence of this document and are not present in
the output.

\var{WindowsVersion} is the ``string name'' of the version of Windows in use,
such as ``XP'', ``Vista'', or ``7''. \var{WindowsEdition} is the edition of the
same; such as ``Home'', ``Professional'', or ``Ultimate''. The values
\var{Major}, \var{Minor}, \var{Build}, and \var{ServicePack} match the current
version information of the operating system in use.

\var{ProcesserArchitecture} is either the string ``\verb|x86|'' or
``\verb|x64|'', matching the installed operating system type. (Note that this is
\textit{not} the capability of the current processor)

Finally, \var{FreeRam} and \var{TotalRam} are the number of Mebibytes of memory
that are available for use by programs on the current running system.

Examples of the 4th line:
\begin{verbatim}
Microsoft Windows Vista Professional N x86 6.0.6000.0 1023/8096 MB Free
Microsoft Windows 7 Ultimate x64 6.1.7601.1 4547/8071 MB Free
\end{verbatim}

\subsubsection{Section}
The heart of the output log is split up into several delimited portions called
``sections''.  Each script action shall emit its own section to the log as

\begin{verbatim}
================ ${Section Name} ===============
\end{verbatim}

That is, the name of the section with one space of padding, centered in a block
of equals (\verb|=|) signs 50 total characters wide. In the case that a tie
exists with respect to the centering, Instalog shall prefer placing the name of
the section farther to the right than to the left.

Beneath the section header, the action can emit different status messages on
separate lines.  The line output format is defined differently for actions that
are scan or repair actions.
\begin{description}
\item[Scanning Action]
Lines that are associated with actions that are defined in
\ref{sec:scanning_actions} can simply have their status message emitted to the log.
\item[Repair Action] \label{repairoutput}
Lines that are associated with actions defined in \ref{sec:repair_actions} must
either succeed or fail after they have been performed.  If the action succeeded,
it shall be emitted to the log as
\vspace{-\baselineskip}
\begin{verbatim}
[ OK ] ${message}
\end{verbatim}
where \var{message} is some information about the action.  Similarly, if the
action failed, it shall be emitted to the log as
\vspace{-\baselineskip}
\begin{verbatim}
[FAIL] ${message}
\end{verbatim}
\end{description}

\subsubsection{Footer}
The footer shall not have a section title. It is a line of the following form
surrounded by newlines:
\begin{verbatim}
Instalog ${version} Finished ${Date}
\end{verbatim}

Here, \var{version} is the version of Instalog used to generate the report, and
\var{Date} is the date when the report was completed using the millisecond date
format defined in section \ref{millidate}.

\subsection{Backup Actions}
An important aspect of Instalog's scripting actions is that most are designed to
be ``safe,'' meaning that changes that are made are designed to be backed up in
some manner so that if something goes wrong, the change can be reverted. 
Several backup procedures are described in this section to provide this.  While
these procedures are in place, this tool will not provide functionality to
restore the backups it takes.  Rather, this will be left up to the user and
other external tools.

Instalog shall use two different backup procedures.  Each script action must
dictate which backup procedure it uses.  
\subsubsection{File Backup} \label{sec:file_backup}
Rather than deleting files, Instalog shall ``quarantine'' them, that is,
Instalog will move them from their current location to some safe location. 
Quarantined files will be placed into a zip container in the corresponding
backup folder for the current script.  This will allow quarantined files to be
easy to associate with a given script run and therefore easy to restore if the
need arises.  

All files quarantined by a script shall be placed in a zip file named
\verb|Files.zip|, which must be located in the backup directory defined in
\ref{sec:backup_folder}.  Files shall be stored in the zip file with the same
path that they have on disk with the exception that the drive letter shall be
the top-most directory.  For example, the file \verb|C:\dir\sub\file.txt| shall
be saved in the zip as \verb|C\dir\sub\file.txt|.
\subsubsection{Registry Backup} \label{sec:reg_backup}
If a script will modify the registry in any way, the registry must first be
backed up in its entirety.  The preferred method for registry backups is to
create them in terms of Microsoft's System Restore feature, creating a  restore
point. Therefore, Instalog shall first attempt to create a restore point.  If
the restore point creation is successful, then this is a sufficient backup.
However, if the restore point creation fails for some reason, then the entire
registry hive must be dumped manually.  Each registry hive shall be copied into
the backup defined in \ref{sec:backup_folder}. Specifically, the registry hives
are defined to be (where \var{UserHive} is one hive per user loaded on the
machine):
\begin{enumerate}
    \item \verb|HKEY_USERS\${UserHive}|
    \item \verb|HKEY_LOCAL_MACHINE\SYSTEM|
    \item \verb|HKEY_LOCAL_MACHINE\SOFTWARE|
    \item \verb|HKEY_LOCAL_MACHINE\SAM|
    \item \verb|HKEY_LOCAL_MACHINE\HARDWARE|
\end{enumerate}

\subsection{Default Script Sections} \label{sec:default_script_sections}
Instalog's standard (that is, run without a script) output consists of the
following sections in the following order:
\begin{enumerate}
    \item Header
    \item Running Processes
    \item Machine PsuedoHJT Report
    \item $n$ User PseudoHJT Reports (One for each loaded user registry on the
    system)
    \item Mozilla Firefox (If Mozilla Firefox is installed)
    \item Google Chrome (If Google Chrome is installed)
    \item Created Last 30
    \item Find3M Report
    \item Event Viewer (If any relevant events need be reported)
    \item Machine Specifications
    \item Restore Points
    \item Installed Programs
    \item Footer
\end{enumerate}

\subsection{Additional Script Sections}
The following additional sections are available but they are not
generated in the default report:
\begin{description}
\item[DNS Check] Displayed if the \verb|:dnscheck| scripting section is used.
\item[Directory] Displayed if the \verb|:dirlook| scripting section is used.
\item[VirusTotal] Displayed if the \verb|:virustotal| scripting section is used.
\item[MRC Upload] Displayed if the \verb|:mrc| scripting section is used.
\item[Process Kill] Displayed if the \verb|:kill| scripting section is used.
\item[File Quarantine] Displayed if the \verb|:move| scripting section is used.
\item[Security Center] Displayed if the \verb|:securitycenter| scripting section
is used.
\item[Registry] Displayed if the \verb|:registry| scripting action is used.
\end{description}
