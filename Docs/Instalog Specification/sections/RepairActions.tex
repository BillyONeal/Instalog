\section{Repair Actions} \label{sec:repair_actions}
This section defines all of the ``repair'' actions that Instalog will support. 
These actions modify the system's configuration in some way and are as such
considered ``system-altering'' actions. \label{systemaltering}

\subsection{Registry Backup}
%Bill

\subsection{Kill Process (\texttt{:KillProcess})}
This action will kill specified process(es) that are listed in the list of
\var{items} defined in \ref{sec:script_syntax}.  This action does not require
any argument.  

For each \var{process} listed in \var{items}, Instalog shall attempt to kill any
and all processes matching the supplied \var{process}.  If the \var{process}
supplies command line arguments, these must be included in the search for
processes.  If all processes matching \var{process} were killed successfully,
then the operation shall be considered a success.  If any one kill operation
fails, then the entire operation shall be considered as unsuccessful.  This
status shall be logged according to \ref{repairoutput}.

\subsection{VirusTotal Upload}
%Bill

\subsection{MRC Upload}
%Bill

\subsection{File Quarantine (\texttt{:Move})}
This simple action will put files into quarantine.  This action does not require
any arguments, but does require a list of \var{items} (\ref{sec:script_syntax})
to move into quarantine.  Each file in \var{items}, shall be placed into
quarantine according to the scheme defined in \ref{sec:file_backup}.

If adding the file to the compressed archive fails, then the tool must not
attempt to delete the file.  If adding the file to the archive fails or deleting
the file fails, then the quarantine action for the file shall be considered
failed.  Otherwise, the action shall be considered a success.  

Each file shall be emitted to the log according to the scheme defined in
\ref{repairoutput}.

\subsection{Hosts File Reset (\texttt{:HostsReset})}
This action resets the hosts file (\verb|%WinDir%\System32\Drivers\Etc\hosts|)
to a default hosts file only containing entry(s) for localhost.  It does not
require any additional argument or item input. First, it shall backup the hosts
file according to the scheme defined in \ref{sec:file_backup}.  If the backup
fails, then the operation shall be emitted to the log as failed according
\ref{repairoutput}. Once the file is backed up, the hosts file shall be replaced
with a simple hosts file.  The host file varies based on the operating system:

\begin{description}
\item[Windows XP or Windows Server 2003 and older] \hfill 
\begin{verbatim}
# Hosts file reset by Instalog

127.0.0.1       localhost
\end{verbatim}
\item[Windows Vista or Windows Server 2008] \hfill 
\begin{verbatim}
# Hosts file reset by Instalog

127.0.0.1       localhost
::1             localhost
\end{verbatim}
\item[Windows 7] \hfill 
\begin{verbatim}
# Hosts file reset by Instalog

# localhost resolution handled within DNS
\end{verbatim}
\end{description}

The status of the write operation shall be written to the log according to
\ref{repairoutput}.  

\subsection{Mozilla Firefox}
%Jacob

\subsection{Google Chrome}
%Jacob

\subsection{Security Center}
%Bill

\subsection{Registry}
%Bill
%32/64 bit
%Remove single MultiSZ value
%ExpandSZ as a string
%MultiSZ as a string
%Remove comma seperated bit like appinitdlls
\subsection{LSP Chain}
%Bill

\subsection{Directory Listing}
%Whoever gets there first

\subsection{DNS Check}
%Whoever gets there first
