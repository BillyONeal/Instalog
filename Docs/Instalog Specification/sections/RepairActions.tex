\section{Repair Actions} \label{sec:repair_actions}
This section defines all of the ``repair'' actions that Instalog will support. 
These actions modify the system's configuration in some way and are as such
considered ``system-altering'' actions. \label{systemaltering}

\subsection{Kill Process (\texttt{:kill})}
This action will kill specified process(es) that are listed in the list of
\var{items} defined in \ref{sec:script_syntax}.  This action does not require
any argument.  

For each \var{process} listed in \var{items}, Instalog shall attempt to kill any
and all processes matching the supplied \var{process}.  If the \var{process}
supplies command line arguments, these must be included in the search for
processes.  If all processes matching \var{process} were killed successfully,
then the operation shall be considered a success.  If any one kill operation
fails, then the entire operation shall be considered as unsuccessful.  This
status shall be logged according to \ref{repairoutput}.

\subsection{VirusTotal Upload (\texttt{:VirusTotal})}
The VirusTotal upload segment shall use the publicly available VirusTotal
upload API (see \url{https://www.virustotal.com/documentation/public-api/}).

One note, the public VirusTotal API is limited to 4 requests per host per
minute, so putting large numbers of files in this section may result in
significant error messages and/or scan failures.

For each item in the set of \var{items} provided to this action as arguments,
Instalog shall upload the file to VirusTotal, and log similar to the following
format:
\begin{verbatim}
${Item} -> ${VirusTotalUrl}
\end{verbatim}
where the \var{VirusTotalUrl} is the URL containing the results of the
VirusTotal check.

\subsection{MRC Upload (\texttt{:mrc})}
BleepingComputer.com operates a file upload service whereby volunteers can
request files from users. Currently, the system operates from the web page
\url{http://www.bleepingcomputer.com/submit-malware.php?channel=##}, where \#\#
is the channel number assigned to the person to whom the file should be sent.

The \var{arg} for this scripting action shall be defined to be the target MRC
channel to which the upload should be sent.

The first item in the set \var{items} shall be the URL to the topic which shall
be associated with the uploads in the MRC system. If the first line cannot be
interpreted as a URL, this step will be skipped, and no URL will be associated
with the uploads.

The remaining \var{items} shall be a set of files to upload to
BleepingComputer.com's MRC system. For each item, Instalog shall indicate
success or failure to upload the file in its log output.

\subsection{File Quarantine (\texttt{:Move})} \label{quarantine}
This simple action will put files into quarantine.  This action does not require
any arguments, but does require a list of \var{items} (\ref{sec:script_syntax})
to move into quarantine.  Each file in \var{items}, shall be placed into
quarantine according to the scheme defined in \ref{sec:file_backup}.

If adding the file to the compressed archive fails, then the tool must not
attempt to delete the file.  If adding the file to the archive fails or deleting
the file fails, then the quarantine action for the file shall be considered
failed.  Otherwise, the action shall be considered a success.  

Each file shall be emitted to the log according to the scheme defined in
\ref{repairoutput}.

\subsection{Hosts File Reset (\texttt{:HostsReset})}
This action resets the hosts file (\verb|%WinDir%\System32\Drivers\Etc\hosts|)
to a default hosts file only containing entry(s) for localhost.  It does not
require any additional argument or item input. First, it shall backup the hosts
file according to the scheme defined in \ref{sec:file_backup}.  If the backup
fails, then the operation shall be emitted to the log as failed according
\ref{repairoutput}. Once the file is backed up, the hosts file shall be replaced
with a simple hosts file.  The host file varies based on the operating system:

\begin{description}
\item[Windows XP or Windows Server 2003 and older] \hfill 
\begin{verbatim}
# Hosts file reset by Instalog

127.0.0.1       localhost
\end{verbatim}
\item[Windows Vista or Windows Server 2008] \hfill 
\begin{verbatim}
# Hosts file reset by Instalog

127.0.0.1       localhost
::1             localhost
\end{verbatim}
\item[Windows 7] \hfill 
\begin{verbatim}
# Hosts file reset by Instalog

# localhost resolution handled within DNS
\end{verbatim}
\end{description}

The status of the write operation shall be written to the log according to
\ref{repairoutput}.  

\subsection{Mozilla Firefox}
Mozilla Firefox has three different fix actions associated with it.  These
actions are described in the following three sections.

Each action will operate on the contents of the default Firefox profile.

If any action is present, then a section shall be emitted to the log as
``Firefox fix" according to the scheme in \ref{logoutput}.  Firefox must not
be running for all of these operations.  If it is running, a dialog box shall
be displayed informing the user that Firefox will be killed.  

\subsubsection{User Preferences (\texttt{:FirefoxFixPreferences})}
\begin{description}
\item[Input] \hfill \\
This action does not require any argument, but optionally will accept additional
input lines.  Input lines can have one of two formats.  The first is simply the
preference and then a newline.  The other format accepts a preference, and then
a space, and then the value the preference shall have (escaped according to
the scheme defined in \ref{urlescape} with \verb|#| as the escape character and
the newline as the right delimiter). 
\item[Operation] \hfill \\
This action must always reset the preferences listed in
\ref{sec:ff_user_preferences}.  To do this, the preferences can simply be
deleted from the preferences file in the default Firefox profile.  Then, the
script shall process any of the optional lines.  If a preference is provided
with no value, then it shall be removed from the preferences file.  If the
preference does not exist, it does not have to be removed.  If there is a value
provided with the input, then the value of the specified preference shall be
set to the unescaped version of the value provided.  If the preference does not
exist in the preferences file, it shall be created.
\item[Output] \hfill \\
The output from this operation shall be emitted to the log according to the
scheme defined in \ref{repairoutput} with \var{message} as ``Preferences reset
successfully'' if the file write succeeds and ``Preferences reset failed'' if
the file write operation fails.
\end{description}

\subsubsection{Extensions and Themes (\texttt{:FirefoxFixExtension})}
This operation shall act the same as the quarantine scheme
defined in \ref{quarantine}. The only reason it needs its own name is because it
requires Firefox not to be running first.

\subsubsection{Plugins (\texttt{:FirefoxFixPlugins})}
This operation shall act the same as the quarantine scheme
defined in \ref{quarantine}. The only reason it needs its own name is because it
requires Firefox not to be running first.

\subsection{Google Chrome}
Google Chrome has three different fix actions associated with it.  These
actions are described in the following three sections.

If any action is present, then a section shall be emitted to the log as
``Chrome fix" according to the scheme in \ref{logoutput}.  Chrome must not
be running for all of these operations.  If it is running, a dialog box shall
be displayed informing the user that Chrome will be killed. 

\subsubsection{Plugins (\texttt{:ChromeFixPlugins})}
This operation shall act the same as the quarantine scheme
defined in \ref{quarantine}. The only reason it needs its own name is because it
requires Chrome not to be running first.

\subsubsection{Extensions (\texttt{:ChromeFixExtensions})}
\begin{description}
\item[Input] \hfill \\
This action does not require any argument, but requires additional input lines. 
Each input line shall simply be the \var{id} for each extension that shall be
removed.  
\item[Operation] \hfill \\
First, the files related to the \var{id} shall be quarantine according to the
scheme defined in \ref{quarantine}.  These files are located in:
\vspace{-\baselineskip}
\begin{verbatim}
%userprofile%\AppData\Local\Google\Chrome\User Data\Default\Extensions\${id}
\end{verbatim}
Then, the extension object matching the given extension \var{id} must be removed
from the preferences file.
\item[Output] \hfill \\
The output for each quarantine operation shall be emitted to the log according
to the scheme defined in \ref{quarantine}.  Lastly, the output from this
operation shall be emitted to the log according to the scheme defined in
\ref{repairoutput} with \var{message} as ``Extensions removed from preferences
file successfully'' if the preferences file write succeeds and ``Failed to
remove extensions from preferences file'' if the file write operation fails.
\end{description} 

\subsubsection{User Preferences (\texttt{:ChromeFixPreferences})}
\begin{description}
\item[Input] \hfill \\
This action does not require any argument, but optionally will accept additional
input lines.  Input lines can have one of two formats.  The first is simply the
preference and then a newline.  The other format accepts a preference, and then
a space, and then the value the preference shall have (escaped according to
the scheme defined in \ref{urlescape} with \verb|#| as the escape character and
the newline as the right delimiter). 
\item[Operation] \hfill \\
This action must always reset the preferences listed in
\ref{sec:chrome_user_preferences}.  To do this, the preferences can simply be
deleted from the preferences file.  Then, the
script shall process any of the optional lines.  If a preference is provided
with no value, then it shall be removed from the preferences file.  If the
preference does not exist, it does not have to be removed.  If there is a value
provided with the input, then the value of the specified preference shall be
set to the unescaped version of the value provided.  If the preference does not
exist in the preferences file, it shall be created.
\item[Output] \hfill \\
The output from this operation shall be emitted to the log according to the
scheme defined in \ref{repairoutput} with \var{message} as ``Preferences reset
successfully'' if the file write succeeds and ``Preferences reset failed'' if
the file write operation fails.
\end{description} 

\subsection{Security Center (\texttt{:seccenter})}
The security center action accepts a list of \var{items} corresponding to
instance GUIDs to delete from the security center registration list. For each
item, Instalog shall print the item, along with a message indicating success or
failure of the deletion of the security center instance GUID from Windows
Management Instrumentation's database.

\subsection{Registry (\texttt{:reg})} \label{sec:registry}
The registry section shall consist of a script similar to regedit's \verb|.REG|
files. What follows describes the syntax used there, along with a few
nonstandard extensions defined by Instalog to make working with the registry
easier. Nonstandard extensions are indicated.

For each operation Instalog takes on behalf of the user, Instalog shall print a
message indicating the action attempted, such as deleting a key, deleting a
value, setting a value, creating a new key, etc., and a message indicating
success or failure of that action.

This action triggers a registry backup.

A registry script consists of the following general format:
\begin{verbatim}
; Comments begin with semicolons and are terminated by a newline

; What follows is a key specification. It begins with a string in square
; brackets:
[${DeleteMinus}${Bitness}${EscapeSpec}${PathToKey}]
; After the key specification, there are zero or more value specifications
; similar to one of the following:
@=${Data} ;@ with no value name is the key's default value
"${ValueName}"=${Data} ; A set of value and value modifications
\end{verbatim}

The various variables above are as follows:
\begin{description}
\item[\var{DeleteMinus}] shall be a minus sign (\verb|-|), or the empty string.
If it is a minus sign, Instalog is to delete the indicated key. Any value
actions specified for the indicated key are ignored if a key deletion is
specified.
\item[\var{Bitness}] is an Instalog extension. It shall be the empty string, in
which case Instalog shall match Regedit's default behavior, or the string
\verb|32:| or \verb|64:|, indicating the 32 bit or 64 bit view of the registry.
(Specifically, \verb|32:| causes the key to be opened using
\verb|KEY_WOW64_32KEY|. \verb|64:| will have no effect, because Instalog runs
natively on x64 machines, but it is included for completeness)
\item[\var{EscapeSpec}] is an Instalog extension, it is either the empty string,
in which case it has no effect, or the string ``\verb|E:|'', indicating that the
\var{PathToKey} variable shall be interpreted using the general escape method
defined in \ref{generalescape}, using an escape character of \verb|#| and a
right delimiter of right square bracket (\verb|]|).
\item[\var{PathToKey}] shall indicate the path in the registry the indicated
script targets. If indicated by \var{EscapeSpec}, this value is escaped as
defined above.
\item[\var{ValueName}] is the name of the value indicated by the value
specification. It is escaped using the general escaping format defined in
\ref{generalescape}, using an escape character of backslash (\verb|\|) and a
right delimiter of quote (\verb|"|).
\item[\var{Data}] a specification of a data action for Instalog to take.
\end{description}

Data actions are one of the following:
\begin{description}
\item[\texttt{-}] The minus character (\verb|-|) indicates that the user wishes
to delete the indicated value from the registry.
\item[\texttt{"\var{String}"}] A string escaped using the general escaping
method defined in \ref{generalescape}, using an escape character of \verb|\|, and a right
delimiter of \verb|"|, indicates that the user wishes to create a value of type
\verb|REG_SZ|, with the content indicated in \var{String}.
\item[\texttt{dword:\var{Number}}] indicates the user wishes to create a value
of type \verb|REG_DWORD|, having the value \var{Number}, which may be specified in
literal or hexadecimal (beginning with \verb|0x|) formats.
\item[\texttt{qword:\var{Number}}] same as \verb|dword|, except a
\verb|REG_QWORD| value.
\item[\texttt{hex:\var{HexDigits}}] indicates that the user wishes to create the
indicated value of type \verb|REG_NONE| using the data provided by
\var{HexDigits}. \var{HexDigits} interprets its input as hexadecimal. Any
characters not interpretable as a hex character (such as commas), are ignored,
excepting the backslash \verb|\|, which indicates that the hex digits
specification shall extend to the following line.
\item[\texttt{hex(\var{Type}):\var{HexDigits}}] indicates that the user wishes
to create a value with the numerical type \var{Type}, where the numerical value
corresponds to one of the following:
\begin{enumerate}
\setcounter{enumi}{-1}
    \item \verb|REG_NONE|
    \item \verb|REG_SZ|
    \item \verb|REG_EXPAND_SZ|
    \item \verb|REG_BINARY|
    \item \verb|REG_DWORD|
    \item \verb|REG_DWORD_BIG_ENDIAN|
    \item \verb|REG_LINK|
    \item \verb|REG_MULTI_SZ|
    \item \verb|REG_RESOURCE_LIST|
    \item \verb|REG_FULL_RESOURCE_DESCRIPTOR|
    \item \verb|REG_RESOURCE_REQUIREMENTS_LIST|
    \item \verb|REG_QWORD|
\end{enumerate}
\item[\texttt{hex(b):\var{HexDigits}}] Same as the above, using a type of
\verb|REG_BINARY|. This is an Instalog extension.
\item[\texttt{expand:"\var{String}"}] This is an Instalog extension. Same as a
normal string, except the type created is \verb|REG_EXPAND_SZ|.
\item[\texttt{multi:"\var{String}"}] This is an Instalog extension. Same as a
normal string, except the type created is \verb|REG_MULTI_SZ|.
\item[\texttt{multiadd:"\var{String}"}] This is an Instalog extension.
\var{String} is escaped using the default escape method defined in \ref{generalescape}, using an
escape character of backslash (\verb|\|), and a right delimiter of quote
(\verb|"|). This action shall read the current contents of the indicated value
as a \verb|REG_MULTI_SZ| value, and add \var{String} from the value if it does
not already exist. If the value is not found or the value is not already of
type \verb|REG_MULTI_SZ|, Instalog shall print an error message.
\item[\texttt{multiminus:"\var{String}"}] This is an Instalog extension.
\var{String} is escaped using the default escape method defined in \ref{generalescape}, using an
escape character of backslash (\verb|\|), and a right delimiter of quote
(\verb|"|). This action shall read the current contents of the indicated value
as a \verb|REG_MULTI_SZ| value, and remove \var{String} from the value if it
exists. If the value is not found or the value is not already of type
\verb|REG_MULTI_SZ|, Instalog shall print an error message.
\item[\texttt{commaadd:"\var{String}"}] This is an Instalog extension. Tries to
interpret the indicated value as a space or comma delimited list of items.
Adds \var{String} from that list, where in the script \var{String} is escaped
using the default escaping method defined in \ref{generalescape}, and writes the
list back to that value using the same delimiter originally encountered in the
value. In the event both spaces and commas are used as delimiters in the source
key, Instalog shall write the entry back using the first found delimiter in the
source key value.
\item[\texttt{commaminus:"\var{String}"}] This is an Instalog extension. Tries
to interpret the indicated value as a space or comma delimited list of items.
Removes \var{String} from that list, where in the script \var{String} is escaped
using the default escaping method defined in \ref{generalescape}, and writes the
list back to that value using the same delimiter originally encountered in the
value. In the event both spaces and commas are used as delimiters in the source
key, Instalog shall write the entry back using the first found delimiter in the
source key value.
\end{description}

\subsection{LSP Chain (\texttt{:lsp})}
The LSP command shall accept a set of \var{items} corresponding to files to
remove from Windows' Winsock2 Layered Service Provider chain database. (note
that this set may be empty). Instalog shall remove the indicated file entries
from the registry, and then repair the LSP chains.

This action triggers a registry backup (see \ref{sec:reg_backup}).

Repairing the LSP chains consists of the following actions:
\begin{itemize}
    \item Enumerating the LSP stack as indicated in \ref{sec:lsp}, and removing
    any entries for which the indicated file no longer exists.
    \item Removing any entries explicitly listed as \var{items} for this action
    to process.
    \item Renumbering the numbered LSP entry keys to return them to consecutive
    order.
    \item Modifying the \verb|Num_Catalog_Entries| values to correspond to the
    actual remaining number of LSP registrations.
\end{itemize}

\subsection{Directory Listing (\texttt{:dirlook})}
The directory listing command accepts a set of \var{items}. For each item in the
set, the item is a directory which shall be enumerated (not recursively), and
displayed using the default file listing format defined in \ref{filelisting}.

\subsection{DNS Check (\texttt{:dnscheck})}
The script action DNS Check shall except a list of domains as \var{items} to
check using the normal DNS checking procedure first outlined in section
(\ref{sec:hjt_dnscheck}). The log output, however, is different. This action
shall not attempt to make judgement calls with respect to whether a domain is
good or bad, and simply prints out all the values generated as part of the
DNSCheck enumeration. Each item in the set of \var{items} generates a line
similar to:
\begin{verbatim}
${Item}->${IP}->${ReverseDNS}
\end{verbatim}
where the values of \var{Item} and \var{ReverseDNS} are as specified in the
PseudoHJT section, and \var{IP} is the primary IP address to which \var{Item}
was resolved.
