\documentclass[letterpaper,12pt]{article}
\usepackage[margin=1in]{geometry}
\usepackage{xltxtra}
\setmainfont[Mapping=tex-text]{Liberation Serif}
\setmonofont[Scale=0.8]{Liberation Mono}
\newcommand{\var}[1]{\texttt{\$\{#1\}}}
\usepackage[colorlinks=false,pdfborder=0 0 0]{hyperref}
\usepackage{graphicx}

\title{Instalog Requirements \\ Document Version 1.0}
\author{
Billy R. O'Neal III (bro4@case.edu) \\
Jacob Snyder (jrs213@case.edu) \\ \\
Case Western Reserve University
}

\begin{document}

\maketitle
\vspace{1in}
\begin{center}
\includegraphics[width=2in, height=2in]{figures/InstalogLogo.png}
\end{center}
\newpage



\tableofcontents
\newpage



\section{Intended Users}
Instalog is designed with three types of target users in mind. These ``user
classes'' are listed in the following sections.

\subsection{Home Users}
For a typical home user, Instalog must not display a complicated interface, and
must make it relatively difficult to misstep and take a wrong action. Few
options need be presented, such as the ability to generate a default report and
the ability to take a given script and run it on a target machine. Complicated
features such as analysis must not be displayed; though they may appear as
options that are, by default, deselected.

\subsection{Administrators}
Administrators are similar to home users in that they are physically working at
a computer being examined, but they are different in that they have the intent of
repairing their own computer or the computer of a client. They wish to see
analysis features and more possible options. Instalog must provide a means for
Administrators to use it's analysis features without manual saving and reloading
of log files.

\subsection{Forum Experts}
Forum Experts help typical end users repair their machines remotely over
self-help forums such as BleepingComputer.com or GeeksToGo.com.
These users work remotely, and likely will never see a given target
machine.
Instalog must produce log formats that are human readable in the vast majority
of cases, but which can be passed through common forum software such as Invision
Power Board, phpBB, or vBulletin without destruction of information.
Unfortunately, this makes common data exchange formats such as JSON and XML
unsuitable. 

Moreover, as obtaining additional information from a machine may
have lead times of several days, Instalog's report must be unambiguous; that is,
no two possible system configurations may produce the same output. Experts can
also benefit from log analysis features. Finally, Experts need to be able to
write simple, human readable scripts to perform actions to fix a user's machine
remotely.

\newpage



\section{Scripting Requirements} \label{scripting}
The main functionality of this tool will be implemented with ``scripts.''  Each
script will specify what actions should be taken by the tool.  For instance, the
default scan in section~\ref{scanning} could be implemented with a script. 
Scripts can also be written by users either by hand or by using the GUI
(section~\ref{gui}).  As such, scripts should be written in human-readable text.
Since scripts will frequently be posted to online forums, the script format must
be designed such that information will not be lost when posting to forums
(whitespace, URLs, unicode characters, etc.).  Not only should scripts be
capable of scanning a machine, this is also the method that Instalog will use
for repairing a user's machine (section~\ref{repair}).

\newpage



\section{Output Requirements} \label{output}
One of the main features of this tool is the logging capability.  The log will
be separated into multiple ``sections,'' where each section has similar
information grouped under it.  There will be a default script of actions that is
provided with the tool that will be performed as the first part of this tool.   
There will also be additional script actions that a script can specify to gather
more targeted information about a system.

\newpage



\section{Scanning Requirements} \label{scanning}
One of the main functionalities of Instalog is its ability to scan a machine to
gather information that will be useful for system diagnostics.  There are a
number of different scan features that the tool will support.  These features
are briefly described below.

\begin{description}
    \item[Running Processes] This will list the currently running processes for
    all users on the machine.
    \item[PsuedoHJT Report] This will implement functionality similar to
    \textit{Hijack This} or \textit{DDS}.  It will run for the machine as well
    as for each user on the system.
    \item[Mozilla Firefox] This will common Firefox infection routes including
    but not limited to extensions, plugins, and preferences.
    \item[Google Chrome] This will common Chrome infection routes including
    but not limited to extensions, plugins, and preferences.
    \item[Services/Drivers] This will list all of the services and drivers on a
    system.
    \item[Created Last 30] This is designed to be the same report as the Created
    Last 30 report from \textit{DDS}, which lists the files from target
    locations that have been updated in the last thirty days.
    \item[Find3M] This is designed to be the same report as the Find3M report
    from \textit{DDS}, which essentially lists ``interesting'' files that have
    been updated in the last three months.
    \item[Event Viewer] This will report any problems in the Windows event log.
    \item[Machine Specifications] This will list important system specifications
    such as processor type, disk partitions, etc.
    \item[Restore Points] This will list all existing restore points.
    \item[Installed Programs] This will list all installed programs.
\end{description}

Each of these options will run in the default scanning script.  Each option
should also be callable from a custom script as described in
section~\ref{scripting}.  All of these actions are designed to be non-invasive,
meaning that they should not modify the user's machine in any way.

\newpage



\section{Repair Action Requirements} \label{repair}
A fix script is a textual representation of the actions that should be taken to
clean up a system.  Most of the log sections listed in the previous section will
have fix actions associated with them.  Again, the specifics about this are too
detailed to include in a document of this scope, but will be explained in full
in the specification document.

One very important requirement of fix script actions is that they MUST create a
backup before they proceed so that the action taken can be reverted.  

\newpage



\section{Graphical User Interface Requirements} \label{gui}
The graphical user interface will be the only method for interfacing with this
tool.  The interface is designed such that it will enable all three of the user
classes described in the ``Intended Users'' section to go through their
appropriate workflows.  As such, the GUI must bridge the gap between being
simple enough for home users to use yet complicated enough for power users to
build complex fix scripts.  This balance is achieved by splitting the GUI up
into several screens for completing various parts of the workflow.

Like the other sections, enumerating all of the requirements and specifications
of this would be beyond the scope of this document.  This being said, the GUI
should include the following screens:

\begin{enumerate}
  \item Main screen
  \item Running screen
  \item Run completed screen
  \item Analysis screen
  \item Analysis complete screen
  \item Finished screen
\end{enumerate}

This screens will be connected in the manner described in Figure 1.

\newpage



\section{Other Requirements}
Instalog will support all Microsoft Windows NT variants released later than
Windows 2000 for x86 and x64 based computers.  Specifically, this tool shall
support:
\begin{itemize}
  \item Windows 2000 (x86, SP4 only)
  \item Windows XP (x86 and x64, RTM, SP1, SP2 and SP3 (on x85 machines))
  \item Windows Vista (x86 and x64, RTM, SP1, and SP2)
  \item Windows 7 (x86 and x64, RTM and SP1)
  \item Windows Server 2003 (x86 and x64, RTM, SP1, SP2)
  \item Windows Server 2003 R2 (x86 and x64, RTM, SP1, and SP2)
  \item Windows Server 2008 (x86 and x64, RTM, SP1, and SP2)
  \item Windows Server 2008 R2 (x64, RTM, and SP1)
\end{itemize}

\newpage



\end{document}
