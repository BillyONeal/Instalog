\documentclass[letterpaper,12pt]{article}
\usepackage[margin=1in]{geometry}
\usepackage{xltxtra}
\setmainfont[Mapping=tex-text]{Liberation Serif}
\setmonofont[Scale=0.8]{Liberation Mono}
\newcommand{\var}[1]{\texttt{\$\{#1\}}}
\usepackage[colorlinks=false,pdfborder=0 0 0]{hyperref}
\usepackage{graphicx}

\title{Instalog Testing and Validation \\ Document Version 1.0}
\author{
Billy R. O'Neal III (bro4@case.edu) \\
Jacob Snyder (jrs213@case.edu) \\ \\
Case Western Reserve University
}

\begin{document}

\maketitle
\vspace{1in}
\begin{center}
\includegraphics[width=2in, height=2in]{figures/InstalogLogo.png}
\end{center}
\newpage



\tableofcontents
\newpage



\section{Introduction}
This document describes the testing and validation processes for the Instalog
tool.  Please note that this document is a work in progress, and is developed as
development progresses on the tool.  

Testing and validation for this tool are extremely important.  Since this tool
makes serious modifications to Windows to remove malware, it is extremely
important that the tool be tested to minimize the likelihood of destroying
users' machines.  Since the current development phase is focused on scanning,
testing isn't \textit{as} important because scanning does not modify the machine
in any way.  However, since many of the components developed to aid in the
scanning features will also be used in the system modification features, it is
still important to test these components extensively.

Once the scanning features are done, this document will be revisited and updated
with the testing and validation plans for the system-modifying features.

\newpage



\section{Testing} \label{testing}
\subsection{Test-Driven Development} \label{test-driven-development}
The main form of testing at this point for the project is unit testing in
test-driven development.  In the development of any code, the tests are written
in conjunction with the implementation.  This allows the code to be modified
with less fear of breaking things.  It also lends itself well to validation
(see section~\ref{validation}).  Currently, this is the only type of testing
done on the project because the components do not interact with each other, so
unit testing each component is sufficient.

At the time of this writing, the test suite for the project contains over two
hundred tests.

\newpage



\section{Validation} \label{validation}
\subsection{Operating System Support Validation}
Validation for the tool is fairly straight-forward at this point.  Once the
scanning sections are implemented and deemed working on the authors' machines,
the test suite from section~\ref{test-driven-development} will be run on all
target operating systems in virtual machines to make sure everything runs
smoothly.  In addition, the output from the scanning component will be examined
to ensure that everything seems normal.  Ideally, this process shouldn't take
long as there isn't much difference between the different operating systems that
this tool will support.

\subsection{Field Validation}
In addition to testing on virtual machines, a beta version of the test suite and
scanning binary will be released to friends and ``forum experts'' in the field. 
The test suite and tool will be run on these machines to make sure that
everything works on ``real world'' machines as opposed to simply the virtual
machines that the authors have access to.  

\newpage


\appendix
\section{License} \label{license}
Instalog itself is to be released under the two clause form of the BSD license,
which is reprinted below:

\begin{verbatim}
Copyright © Jacob Snyder, Billy O'Neal III, and "sUBs"
All rights reserved.

Redistribution and use in source and binary forms, with or without
modification, are permitted provided that the following conditions are met: 

1. Redistributions of source code must retain the above copyright notice, this
   list of conditions and the following disclaimer. 
2. Redistributions in binary form must reproduce the above copyright notice,
   this list of conditions and the following disclaimer in the documentation
   and/or other materials provided with the distribution. 

THIS SOFTWARE IS PROVIDED BY THE COPYRIGHT HOLDERS AND CONTRIBUTORS "AS IS" AND
ANY EXPRESS OR IMPLIED WARRANTIES, INCLUDING, BUT NOT LIMITED TO, THE IMPLIED
WARRANTIES OF MERCHANTABILITY AND FITNESS FOR A PARTICULAR PURPOSE ARE
DISCLAIMED. IN NO EVENT SHALL THE COPYRIGHT OWNER OR CONTRIBUTORS BE LIABLE FOR
ANY DIRECT, INDIRECT, INCIDENTAL, SPECIAL, EXEMPLARY, OR CONSEQUENTIAL DAMAGES
(INCLUDING, BUT NOT LIMITED TO, PROCUREMENT OF SUBSTITUTE GOODS OR SERVICES;
LOSS OF USE, DATA, OR PROFITS; OR BUSINESS INTERRUPTION) HOWEVER CAUSED AND
ON ANY THEORY OF LIABILITY, WHETHER IN CONTRACT, STRICT LIABILITY, OR TORT
(INCLUDING NEGLIGENCE OR OTHERWISE) ARISING IN ANY WAY OUT OF THE USE OF THIS
SOFTWARE, EVEN IF ADVISED OF THE POSSIBILITY OF SUCH DAMAGE.
\end{verbatim}

This document, along with all other documentation related to Instalog,  is to be
released under the Creative Commons Attribution 3.0 Unported license. Human
readable and lawyer readable versions of this license can be found at
\url{http://creativecommons.org/licenses/by/3.0/}.

\newpage

\section{Revision History} \label{revision_history}
\begin{tabular}{| l | l | l |}
\hline
\textbf{Version} & \textbf{Date} & \textbf{Description} \\
\hline
1.0 & March 22, 2012 & Initial document, scanning testing \\
\hline
\end{tabular}



\end{document}
