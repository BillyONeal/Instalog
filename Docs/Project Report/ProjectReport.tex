\documentclass[letterpaper,12pt]{article}
\usepackage[margin=1in]{geometry}
\usepackage{xltxtra}
\setmainfont[Mapping=tex-text]{Liberation Serif}
\setmonofont[Scale=0.8]{Liberation Mono}
\newcommand{\var}[1]{\texttt{\$\{#1\}}}
\usepackage[colorlinks=false,pdfborder=0 0 0]{hyperref}
\usepackage{graphicx}

\title{Instalog Project Report \\
Interim Report 1}
\author{
Billy R. O'Neal III (bro4@case.edu) \\
Jacob Snyder (jrs213@case.edu) \\ \\
Case Western Reserve University
}

\begin{document}

\maketitle
\vspace{1in}
\begin{center}
\includegraphics[width=2in, height=2in]{figures/InstalogLogo.png}
\end{center}
\newpage



\tableofcontents
\newpage



\section{Abstract}
The Microsoft Windows operating system is a complicated system.  When it is
working, everything is fine.  However, when something goes wrong, it is often a
frustrating experience troubleshooting what the problem is, so frequently the
easiest fix is to backup the machine and then reformat it and reinstall Windows.
This approach can severely impact productivity and is complicated for
non-technical users.  

One common source of problems for users of the Windows operating system is
infection from malware.  This has caused there to be a sizable market for
antimalware software.  This software is very effective at removing common
infections because at a high level, the software vendors simply need to maintain
a list of common infections and the safe way to remove them.  Some tools are
even moderately effective at removing unknown threats by using heuristic
techniques.  However, these techniques have limited effectiveness especially
against obscure or new malware.  

Therefore, a demand exists for tools that aid users in removing these obscure
malware infections.  This project, Instalog, is a response to this need. 
Instalog is designed to be a tool with two purposes: information gathering and
system repair.

First, it scans Windows installations to gather as much information as possible
about the machine.  It then filters this information using whitelists to remove
mundane information like system defaults.  It outputs a log of information that
is non-standard on the machine which will hopefully aid an expert in identifying
what the root cause of an infection is.

Second, Instalog provides a capacity to fix infections.  Given a log output,
experts can generate scripts to fix an infection either by hand or by using the
GUI.  Instalog can run these scripts to remove an infection.  

Instalog is designed such that it can support three different ``user classes.'' 
The first is that of a ``home user'' that has an infected machine but doesn't
know how to fix it.  This user can use Instalog to scan their machine and then
post it online to get assistance.  Here, an ``expert'' can download the ``home
user's'' log, analyze it, and produce a fix script for them to run.  The ``home
user'' can then download this script and run it using Instalog to fix their
machine.  This process may involve several iterations.  The other user case is
that of a ``system administrator'' that plays both roles -- ``system
administrators'' use Instalog to scan their machines and know enough about the
machine to generate and run a suitable fix script.  

\newpage



\section{Introduction}
\subsection{Background}
% TODO Bill?  Otherwise, I'll just add some details to Abstract

\subsection{Related Work}
Instalog is inspired by several similar tools which all share some basic
functionality.  In many ways, Instalog can be viewed as an evolution of
these tools:

\begin{itemize}
    \item TrendMicro's {\em Hijack This} (HJT)
    \item ``sUBs'' {\em Doesn't Do Squat} (DDS)
    \item ``random/random'''s {\em Random's System Information Tool} (RSIT)
    \item ``OldTimer'''s {\em OTA}, {\em OTS}, and {\em OTL} (formerly
    OTAnalyzeIt, OTScanIt, and OTListIt, respectively)
    \item Sysinternals' {\em Autoruns}
    \item Runscanner's {\em Runscanner}
\end{itemize}

All of these tools purport to accomplish similar goals to Instalog. However,
each of these tools has bugs or specific behavior which cause problems for at
least one of Instalog's three intended user groups.  Specifically, the above
tools contain one or more of the above problems:

\begin{itemize}
    \item Incorrect handling and escaping of log data
    \item Lack of published specifications, documentation, or source code
    \item Outstanding bugs that the authors are unwilling or unable to fix
    \item Lack of scriptability, for the purposes of modifying log output and
    malware removal.
    \item Lack of 64 bit support.
    \item Lack of Unicode support.
    \item Lack of enumeration of some types of useful log information.
\end{itemize}

Instalog will attempt to solve those problems by combining characteristics of
the above tools which are deemed useful while mixing in a few tricks of its
own.

\subsection{Project Description}
Project description (what it is, goals, technical issues, etc.) in more
technical detail than the abstract; it should include a detailed description of
the problem,

\subsection{Progress}
As of this writing, the majority of the work has been planning and documentation
(primarily in writing the specifications).  Since the first use of this tool is
for information gathering, several weeks of time had to be spent in carefully
identifying where the information would be gathered from, how it would be
displayed, and how it would be whitelisted.  Since the second use of this tool
is for fixing machines, the fix script specification had to be designed
incredibly carefully, which took about a week.  The work done in this respect is
found in the documents referenced in section~\ref{software_requirements} and
section~\ref{software_specifications}.  
% TODO Update with any further items.  Also make sure the references differ

For more information, please refer to section~\ref{project_progress}.

\subsection{Report Structure}
% TODO Write this section once report is complete

\newpage



\section{Application}
Instalog can be seen as having three different distinct components:

\begin{enumerate}
  \item Scanning
  \item Scripting and Repairing
  \item GUI
\end{enumerate}

Each of these components will be expanded on in the following subsections.

\subsection{Scanning}
Instalog's scanning abilities are the information-gathering and whitelist
capabilities that will be provided to the user.  Instalog will scan a number of
different types of information such as registry keys, browser settings, and
recently modified files.  These different information points are picked because
they are primarily ``loading points,'' that is, they are locations that malware
would need to modify to be loaded.  Each information point will be filtered
using whitelists that will remove default values and values that are known to be
safe.  What will be left is information that is unique to the user's computer,
hopefully some of which provides information about the malware.

A big challenge with this is the sheer amount of data that is being included in
the scan.  The Instalog project team plan on getting around this by writing
meticulously detailed requirements and specifications documents to make sure
each behavior is well defined before starting to implement the project.
Another challenge with this is making the scan run quickly.  The biggest
challenge is to simply test the scanning behavior.  Since it is impossible to
anticipate every possible machine configuration, the software must be robust
enough to handle varying scenarios across all of the targeted versions of
Windows.

\subsection{Scripting and Repairing}
The next component is Instalog's ability to run scripts and repair systems. 
Instalog will define a scripting language that enables common actions such as
process killing, registry actions, and file quarantine.  It will also provide
scripting actions to run all of the available scanning actions.  This will allow
a script to gather more information about a system if necessary.  

The biggest challenge to this is similar to the scanning challenge -- it will be
very difficult to thoroughly test this.  It is \textit{imperative} that this
functionality is tested extensively though because one mistake in this component
could lead to data loss on a user's machine.  

\subsection{GUI}
The final component of Instalog is the graphical user interface (GUI).  This
interface will allow users to run scans, run scripts, and build scripts based on
scan outputs.  The challenge here is building the GUI using the native GUI
controls, especially since some of the proposed screens such as the script
builder are incredibly complex.  Additionally, the script builder must be tested
heavily as well, because if a poor script is built, this could also lead to data
loss on a user's machine.  

\newpage



\section{Methodology}
This section should discuss any data structure design/maintenance problems, or
algorithmic problems/challenges, and how they will be/are solved; time and space
complexity analysis (if needed) of your algorithms, etc.

\newpage



\section{Software Design}
This section will describe the progress towards the usual software design
information, with likely components on

\begin{itemize}
  \item Application Software Requirements
  \item Application Software Specifications
  \item Software Architecture (i.e., client-server architecture; three-tiered
  design, etc.)
  \item Design Document
\end{itemize}

\label{software_requirements}
\label{software_specifications}
\label{software_design}

\newpage



\section{Project Management and Administrative Details}
Design/update your Gantt chart to show any changes in milestones, timetable, or
responsibility (especially if there are multiple team members). Be sure to
properly indicate who is actually performing each task and the degree of
completion of each task.

Describe how well the project management plan is working. Were there unforeseen
problems with any of the tasks that you have had to work around? Are there
delays in obtaining parts or software? Were major changes to the management plan
or back-up plans required and implemented? What work-arounds were necessary to
keep the project on schedule? It is very important that you describe any changes
in the project plan since the proposal and the reasons for them.
Finally, use your management plan to carefully think about what can be
realistically done by your team between now and the end of the semester.

\newpage



\section{User Interface}
This section contains the user interface of the application. It should have
numbered figures with titles and screenshots of the GUI components, as well as
explanations.

\newpage



\section{Testing and Evaluation}
This section should discuss any testing and evaluation done so far.

\newpage



\section{Project Progress} \label{project_progress}
This section should summarize the progress (or the lack of it) so far, and
address any issues that have arisen recently.

\newpage



\section{Discussion and Conclusions}
These conclusions are not really conclusions since your project is not yet
finished. In this section you should discuss whether the project is on schedule
for completion at the end of the semester. Are there major problems which
require a reevaluation of the project results?  If not, what is the new schedule
and what do you plan to have done by the end of the semester?  You can also
comment on things which have worked better than expected or proved easier to
solve.

\newpage



\section{References}
Any references to the literature, web sites, etc., should follow the ACM
citation standards (Just look at the ACM publications).

\newpage



\section{Licensing}
Instalog itself is to be released under the two clause form of the BSD license,
which is reprinted below:

\begin{verbatim}
Copyright © 2012, Jacob Snyder, Billy O'Neal III, and "sUBs"
All rights reserved.

Redistribution and use in source and binary forms, with or without
modification, are permitted provided that the following conditions are met: 

1. Redistributions of source code must retain the above copyright notice, this
   list of conditions and the following disclaimer. 
2. Redistributions in binary form must reproduce the above copyright notice,
   this list of conditions and the following disclaimer in the documentation
   and/or other materials provided with the distribution. 

THIS SOFTWARE IS PROVIDED BY THE COPYRIGHT HOLDERS AND CONTRIBUTORS "AS IS" AND
ANY EXPRESS OR IMPLIED WARRANTIES, INCLUDING, BUT NOT LIMITED TO, THE IMPLIED
WARRANTIES OF MERCHANTABILITY AND FITNESS FOR A PARTICULAR PURPOSE ARE
DISCLAIMED. IN NO EVENT SHALL THE COPYRIGHT OWNER OR CONTRIBUTORS BE LIABLE FOR
ANY DIRECT, INDIRECT, INCIDENTAL, SPECIAL, EXEMPLARY, OR CONSEQUENTIAL DAMAGES
(INCLUDING, BUT NOT LIMITED TO, PROCUREMENT OF SUBSTITUTE GOODS OR SERVICES;
LOSS OF USE, DATA, OR PROFITS; OR BUSINESS INTERRUPTION) HOWEVER CAUSED AND
ON ANY THEORY OF LIABILITY, WHETHER IN CONTRACT, STRICT LIABILITY, OR TORT
(INCLUDING NEGLIGENCE OR OTHERWISE) ARISING IN ANY WAY OUT OF THE USE OF THIS
SOFTWARE, EVEN IF ADVISED OF THE POSSIBILITY OF SUCH DAMAGE.
\end{verbatim}

This document, along with all other documentation related to Instalog,  is to be
released under the Creative Commons Attribution 3.0 Unported license. Human
readable and lawyer readable versions of this license can be found at
\url{http://creativecommons.org/licenses/by/3.0/}.

\newpage



\section{Appendices}
Use appendices as appropriate. The parts that are necessary are colored in red.
This section can also have, if any, descriptions about background components to
the system and their design:

\begin{itemize}
  \item User Manual.  This is a manual for users to use the system, describing,
  if any, rules and procedures.
  \item Programmers Manual.  This is a manual for a programmer to take your
  code, install it, use it, understand the code, and revise it if necessary. 
  \item Third-party software use and its incorporation to the project.  Examples
  may be graph drawing software, visualization software, XML-related software,
  etc.
  \item Any figures, drawings, and anything else which is too detailed and/or
  too long to put anywhere else in the report. 
  \item License information
\end{itemize}

\newpage



\end{document}
