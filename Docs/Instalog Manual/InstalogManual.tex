\documentclass[letterpaper,12pt]{article}
\usepackage[margin=1in]{geometry}
\usepackage{xltxtra}
\setmainfont[Mapping=tex-text]{Liberation Serif}
\setmonofont[Scale=0.8]{Liberation Mono}
\newcommand{\var}[1]{\texttt{\$\{#1\}}}
\usepackage[colorlinks=false,pdfborder=0 0 0]{hyperref}
\usepackage{graphicx}

\title{Instalog Users Manual \\ Document Version 1.0}
\author{
Billy R. O'Neal III (bro4@case.edu) \\
Jacob Snyder (jrs213@case.edu) \\ \\
Case Western Reserve University
}

\begin{document}

\maketitle
\vspace{1in}
\begin{center}
\includegraphics[width=2in, height=2in]{figures/InstalogLogo.png}
\end{center}
\newpage



\tableofcontents
\newpage



\section{Using Instalog}
Running Instalog is really easy!  Once you have the tool downloaded, just double
click it to run it.  If you are Windows Vista or Windows 7, a window may pop up
asking your permission to run the tool.  This is because Instalog requires
administrator privileges to run because it accesses sensitive portions of the
system.  Just say ``Yes'' in this window.  Instalog will then run and display a
black command prompt window indicating its progress.  Once it is done scanning
your system, it will create a file called ``\verb|Instalog.txt|'' in the same
place that you saved the Instalog executable to.  This contains the results from
the scan.  This is what you would post to forums or other help channels for
assistance.  

If you are interested in interpreting the log output yourself, all of the
features are well documented in the Instalog Specification document.  Read
through the relevant portions of the document to learn how to interpret the log.

\newpage



\section{Reporting Problems}
Instalog is still under development, and as such, may still have issues in it. 
If you encounter unusual behavior while running the tool, please head to the
Instalog website and report a bug.  In the bug report, please include the log
output as well as any other relevant information that may help us replicate the
issues you are having.

\newpage



\appendix
\section{License} \label{license}
Instalog itself is to be released under the two clause form of the BSD license,
which is reprinted below:

\begin{verbatim}
Copyright © 2012, Jacob Snyder, Billy O'Neal III, and "sUBs"
All rights reserved.

Redistribution and use in source and binary forms, with or without
modification, are permitted provided that the following conditions are met: 

1. Redistributions of source code must retain the above copyright notice, this
   list of conditions and the following disclaimer. 
2. Redistributions in binary form must reproduce the above copyright notice,
   this list of conditions and the following disclaimer in the documentation
   and/or other materials provided with the distribution. 

THIS SOFTWARE IS PROVIDED BY THE COPYRIGHT HOLDERS AND CONTRIBUTORS "AS IS" AND
ANY EXPRESS OR IMPLIED WARRANTIES, INCLUDING, BUT NOT LIMITED TO, THE IMPLIED
WARRANTIES OF MERCHANTABILITY AND FITNESS FOR A PARTICULAR PURPOSE ARE
DISCLAIMED. IN NO EVENT SHALL THE COPYRIGHT OWNER OR CONTRIBUTORS BE LIABLE FOR
ANY DIRECT, INDIRECT, INCIDENTAL, SPECIAL, EXEMPLARY, OR CONSEQUENTIAL DAMAGES
(INCLUDING, BUT NOT LIMITED TO, PROCUREMENT OF SUBSTITUTE GOODS OR SERVICES;
LOSS OF USE, DATA, OR PROFITS; OR BUSINESS INTERRUPTION) HOWEVER CAUSED AND
ON ANY THEORY OF LIABILITY, WHETHER IN CONTRACT, STRICT LIABILITY, OR TORT
(INCLUDING NEGLIGENCE OR OTHERWISE) ARISING IN ANY WAY OUT OF THE USE OF THIS
SOFTWARE, EVEN IF ADVISED OF THE POSSIBILITY OF SUCH DAMAGE.
\end{verbatim}

This document, along with all other documentation related to Instalog,  is to be
released under the Creative Commons Attribution 3.0 Unported license. Human
readable and lawyer readable versions of this license can be found at
\url{http://creativecommons.org/licenses/by/3.0/}.

\newpage

\section{Revision History} \label{revision_history}
\begin{tabular}{| l | l | l |}
\hline
\textbf{Version} & \textbf{Date} & \textbf{Description} \\
\hline
1.0 & April 21, 2012 & Initial document \\
\hline
\end{tabular}



\end{document}
