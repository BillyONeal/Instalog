\documentclass[letterpaper,12pt]{article}
\usepackage[margin=1in]{geometry}
\usepackage{xltxtra}
\setmainfont[Mapping=tex-text]{Liberation Serif}
\setmonofont[Scale=0.8]{Liberation Mono}
\newcommand{\var}[1]{\texttt{\$\{#1\}}}
\usepackage[colorlinks=false,pdfborder=0 0 0]{hyperref}
\usepackage{graphicx}

\title{Instalog Design Document \\ Document Version 1.0}
\author{
Billy R. O'Neal III (bro4@case.edu) \\
Jacob Snyder (jrs213@case.edu) \\ \\
Case Western Reserve University
}

\begin{document}

\maketitle
\vspace{1in}
\begin{center}
\includegraphics[width=2in, height=2in]{figures/InstalogLogo.png}
\end{center}
\newpage



\tableofcontents
\newpage



\section{Introduction}
This document explains how the Instalog tool will be implemented.  This document
is a work in progress and will be updated as features of the tool are
approached.  Since the scanning features are being developed currently, they are
the only features currently designed in this document.  However, since the
scanning features depend on things like scripting and rudimentary user
interfaces, these are also discussed in this document.

\newpage

\section{General Structure}
The general structure to the design of this tool is that there are three
``layers'': Utilities, then Scripting, and lastly User Interfaces.  These are
built in an expanding fashion, meaning that Utilities are independent, then
Scripting is built on top of the Utilities (but the Utilities are not dependent
on Scripting), and the User Interfaces are built upon Scripting and Utilities
(but the Utilities and Scripting are not dependent on the User Interfaces).

\subsection{Utilities}
There isn't much to design here.  The Utilities are common helper functions that
most components of the tool will use, such as string escaping formats and other
parts of the Specification that are referred to frequently.  

\subsection{Scripting}
Scripting is the core of the code for the tool.  Since all the various
operations of the tool can be thought of as scripts, this is the complex design.
However, since script sections don't interact, it's still fairly
straightforward.  

\subsubsection{Sections}
Each part of scripts are split into different ``sections'' which correspond to
the different scanning and fix capabilities described in the specifications
document.  

A section definition is defined as follows (with a sample implementing section
class):
\begin{center}\includegraphics{figures/ISectionDefinition.png}\end{center}

\verb|GetScriptCommand| returns the name that the section is called in scripts. 
\verb|GetName| returns the friendly human readable name of the section. 
\verb|GetPriority| returns the priority of each script section.  The return
value is an enum \verb|LogSectionPriorities| which correspond to the different
priorities in the specifications document.  Lastly, \verb|Execute| takes a
stream to print to, a \verb|ScriptSection|, and option lines. 
\verb|ScriptSection| is defined as follows, and essentially pairs arguments to
section calls in the script.
\begin{center}\includegraphics{figures/ScriptSection.png}\end{center}

\subsubsection{Script}
\verb|Script| is defined as follows:
\begin{center}\includegraphics{figures/Script.png}\end{center}

All of the methods here are self explanatory.  In \verb|Run|, the \verb|ui|
argument is the user interface for the script to report to.  This is the only
dependency in this layer to the UI layer.

Scripts are built with the \verb|ScriptDispatcher|:
\begin{center}\includegraphics{figures/ScriptDispatcher.png}\end{center}
The dispatcher will essentially parse a script, reorder the sections, and merge
applicable intermediate sections.  It will then return a script that can be
executed.

\subsection{User Interfaces}
All User Interfaces will be defined as follows:
\begin{center}\includegraphics{figures/IUserInterface.png}\end{center}
A User Interface is essentially an object that exposes various callbacks for the
script to call as a script progresses.  User Interfaces can be more complicated,
but in this case, a sample Console User Interface is shown to implement the
interface.  

\newpage


\appendix
\section{License} \label{license}
Instalog itself is to be released under the two clause form of the BSD license,
which is reprinted below:

\begin{verbatim}
Copyright © Jacob Snyder, Billy O'Neal III, and "sUBs"
All rights reserved.

Redistribution and use in source and binary forms, with or without
modification, are permitted provided that the following conditions are met: 

1. Redistributions of source code must retain the above copyright notice, this
   list of conditions and the following disclaimer. 
2. Redistributions in binary form must reproduce the above copyright notice,
   this list of conditions and the following disclaimer in the documentation
   and/or other materials provided with the distribution. 

THIS SOFTWARE IS PROVIDED BY THE COPYRIGHT HOLDERS AND CONTRIBUTORS "AS IS" AND
ANY EXPRESS OR IMPLIED WARRANTIES, INCLUDING, BUT NOT LIMITED TO, THE IMPLIED
WARRANTIES OF MERCHANTABILITY AND FITNESS FOR A PARTICULAR PURPOSE ARE
DISCLAIMED. IN NO EVENT SHALL THE COPYRIGHT OWNER OR CONTRIBUTORS BE LIABLE FOR
ANY DIRECT, INDIRECT, INCIDENTAL, SPECIAL, EXEMPLARY, OR CONSEQUENTIAL DAMAGES
(INCLUDING, BUT NOT LIMITED TO, PROCUREMENT OF SUBSTITUTE GOODS OR SERVICES;
LOSS OF USE, DATA, OR PROFITS; OR BUSINESS INTERRUPTION) HOWEVER CAUSED AND
ON ANY THEORY OF LIABILITY, WHETHER IN CONTRACT, STRICT LIABILITY, OR TORT
(INCLUDING NEGLIGENCE OR OTHERWISE) ARISING IN ANY WAY OUT OF THE USE OF THIS
SOFTWARE, EVEN IF ADVISED OF THE POSSIBILITY OF SUCH DAMAGE.
\end{verbatim}

This document, along with all other documentation related to Instalog,  is to be
released under the Creative Commons Attribution 3.0 Unported license. Human
readable and lawyer readable versions of this license can be found at
\url{http://creativecommons.org/licenses/by/3.0/}.

\newpage

\section{Revision History} \label{revision_history}
\begin{tabular}{| l | l | l |}
\hline
\textbf{Version} & \textbf{Date} & \textbf{Description} \\
\hline
1.0 & March 22, 2012 & Initial document \\
\hline
\end{tabular}



\end{document}
